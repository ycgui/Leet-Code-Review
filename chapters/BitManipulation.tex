\chapter{Bit Manipulation}


\section{Sum of Two Integers (E)}
Calculate the sum of two integers a and b, but you are not allowed to use the operator + and -. \\

Example:\\
Given a = 1 and b = 2, return 3. \\

\begin{lstlisting}
class Solution {
public:
    int getSum(int a, int b) {
        int sum = a;
        while (b != 0) {
            sum = a ^ b;                // use ^ to find different bits
            b = (a & b) << 1;           // use & to find carry, then shift one position left
            a = sum;
        }
        return sum;
    }
};
\end{lstlisting}


\section{Bitwise AND of Numbers Range (M)}
Given a range [m, n] where $0 <= m <= n <= 2147483647$, return the bitwise AND of all numbers in this range, inclusive.\\

For example, given the range [5, 7], you should return 4. \\

\begin{lstlisting}
class Solution {
public:
    int rangeBitwiseAnd(int m, int n) {
        int count = 0;
        while (m != n) {
            m >>= 1;
            n >>= 1;
            ++count;
        }
        return m << count;
    }
};
\end{lstlisting}


\section{Grey Code (M)}
The gray code is a binary numeral system where two successive values differ in only one bit.\\

Given a non-negative integer n representing the total number of bits in the code, print the sequence of gray code. A gray code sequence must begin with 0.\\

For example, given n = 2, return [0,1,3,2]. Its gray code sequence is:\\
00 - 0\\
01 - 1\\
11 - 3\\
10 - 2\\

\begin{lstlisting}
// Binary to grey code
class Solution {
public:
    vector<int> grayCode(int n) {
        vector<int> res;
        for (int i = 0; i < pow(2,n); ++i) {
            res.push_back((i >> 1) ^ i);
        }
        return res;
    }
};
\end{lstlisting}


\section{Repeated DNA Sequences (M)}
All DNA is composed of a series of nucleotides abbreviated as A, C, G, and T, for example: "ACGAATTCCG". When studying DNA, it is sometimes useful to identify repeated sequences within the DNA.\\

Write a function to find all the 10-letter-long sequences (substrings) that occur more than once in a DNA molecule.\\

For example,
Given s = "AAAAACCCCCAAAAACCCCCCAAAAAGGGTTT",
Return: ["AAAAACCCCC", "CCCCCAAAAA"].\\

\begin{lstlisting}

\end{lstlisting}


\section{Power of Two (E)}
Given an integer, write a function to determine if it is a power of two. \\
 
\begin{lstlisting}
class Solution {
public:
    bool isPowerOfTwo(int n) {      // 2^x = n
        return (n > 0) && (n & (n-1)) == 0;
    }
};
\end{lstlisting}


\section{Power of Three (E)}
Given an integer, write a function to determine if it is a power of three. \\
 
\begin{lstlisting}
class Solution {
public:
    bool isPowerOfThree(int n) {
        if (n <= 0)
            return false;
            
        while(n % 3 == 0)
            n /= 3;
            
        return n == 1;
    }
};
\end{lstlisting}


\section{Power of Four (E)}
Given an integer (signed 32 bits), write a function to check whether it is a power of 4. \\

\begin{lstlisting}
class Solution {
public:
    bool isPowerOfFour(int num) {
        return (num > 0) && ((num & (num - 1)) == 0) && ((num & 0x55555555) == num);
    }
};
\end{lstlisting}


\section{Number of 1 Bits (E)}
Write a function that takes an unsigned integer and returns the number of "1" bits it has (also known as the Hamming weight).\\

For example, the 32-bit integer "11" has binary representation 00000000000000000000000000001011, so the function should return 3.\\

\begin{lstlisting}
class Solution {
public:
    int hammingWeight(uint32_t n) {
        int res = 0;
        while (n != 0) {
            if (n & 1 == 1)
                ++res;
            n = n >> 1;
        }
        return res;
    }
};
\end{lstlisting}


\section{Counting Bits (M)}
Given a non negative integer number num. For every numbers i in the range $0 \leq i \leq num$ calculate the number of 1's in their binary representation and return them as an array.\\

Example:\\
For num = 5 you should return [0,1,1,2,1,2]. \\

\begin{lstlisting}
class Solution {
public:
    vector<int> countBits(int num) {
        vector<int> res = {0};
        int offset = 2;
        
        for (int i = 1; i <= num; ++i) {                
            if (i >= offset)
                offset *= 2;
            res.push_back(res[i - offset/2] + 1);       // dp[index] = dp[index - offset] + 1;
        }
        
        return res;
    }
};
\end{lstlisting}


\section{Reverse Bits (E)}
Reverse bits of a given 32 bits unsigned integer.\\

For example, given input 43261596 (represented in binary as 00000010100101000001111010011100), return 964176192 (represented in binary as 00111001011110000010100101000000).\\

Follow up:\\
If this function is called many times, how would you optimize it? \\

\begin{lstlisting}
class Solution {
public:
    uint32_t reverseBits(uint32_t n) {
        int res = 0;
        for (int i = 0; i < 32; ++i){
            res += n & 1;       // assign the right most bit of n to res
            n >>= 1;            // remove the assigned bit from n
            if (i < 31)
                res <<= 1;      // move the assigned bit in res to left for reversing
        }
        return res;
    }
};
\end{lstlisting}


\section{Single Number (E)}
Given an array of integers, every element appears twice except for one. Find that single one.\\

Note: Your algorithm should have a linear runtime complexity. Could you implement it without using extra memory? \\

\begin{lstlisting}
class Solution {
public:
    int singleNumber(vector<int>& nums) {
        int result = 0;
        for (int i = 0; i < nums.size(); ++i)
            result ^= nums[i];
        return result;
    }
};
\end{lstlisting}


\section{Single Number II (M)}
Given an array of integers, every element appears three times except for one. Find that single one. \\

Note: Your algorithm should have a linear runtime complexity. Could you implement it without using extra memory? \\

\begin{lstlisting}
class Solution {
public:
    int singleNumber(vector<int>& nums) {
        int x1 = 0;   
        int x2 = 0; 
        int mask = 0;

        for (int i = 0; i < nums.size(); ++i) {
            x2 ^= x1 & nums[i];
            x1 ^= nums[i];
            mask = ~(x1 & x2);
            x2 &= mask;
            x1 &= mask;
     }

     return x1;
    }
};
\end{lstlisting}


\section{Single Number III (M)}
Given an array of numbers nums, in which exactly two elements appear only once and all the other elements appear exactly twice. Find the two elements that appear only once. \\

For example: Given nums = [1, 2, 1, 3, 2, 5], return [3, 5].\\

Note:
    The order of the result is not important. So in the above example, [5, 3] is also correct.\\
    Your algorithm should run in linear runtime complexity. Could you implement it using only constant space complexity?\\

\begin{lstlisting}
class Solution {
public:
    vector<int> singleNumber(vector<int>& nums) {
        int r = 0, n = nums.size(), i = 0, last = 0;
        vector<int> ret(2, 0);

        for (i = 0; i < n; ++i) 
            r ^= nums[i];           // r = A ^ B

        last = r & (~(r - 1));      // get the last '1'
        
        for (i = 0; i < n; ++i)
        {
            if ((last & nums[i]) != 0)
                ret[0] ^= nums[i];          // group with the same position of '1'
            else
                ret[1] ^= nums[i];          // group without the same position of '1'
        }

        return ret;
    }
};
\end{lstlisting}


\section{Binary Clock (Google Interview 8.15.2016)}
Suppose you have a binary clock that consists of two rows of LEDs. The first row has four LEDs representing four binary digits to indicate the hour. The bottom row has six LEDs to represent six binary digits indicating the minute. Write a program to list all the times that consist of exactly three lights being on, and the rest being off. List the times in human readable form.\\

\begin{lstlisting}
class BinaryClock {
public:
    void clockTime() {
        vector<string> res;
        for (int h = 0; h <= 12; ++h) {
            for (int m = 0; m < 60; ++m) {
                if (countBits(h) + countBits(m) == 3) {
                    res.push_back(to_string(h) + ':' + to_string(m));
                }
            }
        }
    }
    int countBits(int value){
      int mask = 1;
      int cnt = 0;
      for(int i = 0; i < 6; i++) {
         if((value & mask) == 1) ++cnt; // find 1, increase cnt
         mask <<= 1; // shift left to find all 1's in value
      }
      return cnt;       
    }
};
\end{lstlisting}

