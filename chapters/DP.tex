\chapter{Dynamic Programming}
\section{Gas Station (M)}
There are N gas stations along a circular route, where the amount of gas at station i is gas[i]. You have a car with an unlimited gas tank and it costs cost[i] of gas to travel from station i to its next station (i+1). You begin the journey with an empty tank at one of the gas stations. Return the starting gas station's index if you can travel around the circuit once, otherwise return -1.\\

Note:
The solution is guaranteed to be unique. \\

\begin{lstlisting}
// Solution 1
class Solution {
public:
    int canCompleteCircuit(vector<int> &gas, vector<int> &cost) {
        int total = 0, sum = 0, start = 0;
        for (int i = 0; i < gas.size(); ++i) {
            total += gas[i] - cost[i];
            sum += gas[i] - cost[i];
            if (sum < 0) {
                start = i + 1;
                sum = 0;
            }
        }
        if (total < 0) return -1;
        else return start;
    }
};

// Solution 2
class Solution {
public:
    int canCompleteCircuit(vector<int>& gas, vector<int>& cost) {
        int total_gas = 0, total_cost = 0, tank = 0, index = 0;
        // If car starts at A and can not reach B, any station between A and B can not reach B
        for (int i = 0; i < gas.size(); ++i) {  
            total_gas += gas[i];                
            total_cost += cost[i];
            tank += gas[i] - cost[i];
            if (tank < 0) {
                index = i + 1;
                tank = 0;
            }
        }
        // If the total number of gas is bigger than the total number of cost, there must be a solution
        if (total_gas < total_cost) return -1;                  
        else return index;
    }
};
\end{lstlisting}


\section{Combination Sum IV (M)}
Given an integer array with all positive numbers and no duplicates, find the number of possible combinations that add up to a positive integer target.\\

Example: nums = [1, 2, 3], target = 4\\
The possible combination ways are:\\
(1, 1, 1, 1)
(1, 1, 2)
(1, 2, 1)
(1, 3)
(2, 1, 1)
(2, 2)
(3, 1)\\

Note that different sequences are counted as different combinations. Therefore the output is 7.\\

Follow up:\\
What if negative numbers are allowed in the given array?\\
How does it change the problem?\\
What limitation we need to add to the question to allow negative numbers? \\

\begin{lstlisting}
class Solution {
public:
    int combinationSum4(vector<int>& nums, int target) {
        vector<int> dp(target+1);
        dp[0] = 1;
        for (int i = 1; i <= target; ++i) {
            for (auto a : nums) {
                if (i >= a) dp[i] += dp[i-a];
            }
        }
        return dp.back();
    }
};
\end{lstlisting}


\section{Perfect Squares (M)}
Given a positive integer n, find the least number of perfect square numbers (for example, 1, 4, 9, 16, ...) which sum to n.\\

For example, given n = 12, return 3 because 12 = 4 + 4 + 4; given n = 13, return 2 because 13 = 4 + 9. \\

\begin{lstlisting}
class Solution {
public:
    int numSquares(int n) {
        vector<int> dp(n + 1, INT_MAX);
        dp[0] = 0;
        
        // if x = a + b * b, the least number of perfect square numbers which sum to x is dp[x], then
        // case1: dp[x] = dp[a] + 1, because b * b is a perfect square number
        // case2: dp[x] = dp[a + b*b], because a + b * b is a perfect square number
        // dp[x] = min(case1, case2)
        for (int i = 0; i <= n; ++i) {
            for (int j = 1; i + j * j <= n; ++j) {
                dp[i + j * j] = min(dp[i + j * j], dp[i] + 1);
            }
        }
        return dp[n];
    }
};
\end{lstlisting}


\section{Coin Change (M)}
You are given coins of different denominations and a total amount of money amount. Write a function to compute the fewest number of coins that you need to make up that amount. If that amount of money cannot be made up by any combination of the coins, return -1.\\

Example 1:
coins = [1, 2, 5], amount = 11
return 3 (11 = 5 + 5 + 1)\\

Example 2:
coins = [2], amount = 3
return -1.\\

Note:
You may assume that you have an infinite number of each kind of coin. \\

\begin{lstlisting}
// Non-recursion
class Solution {
public:
    int coinChange(vector<int>& coins, int amount) {
        vector<int> dp(amount + 1, amount + 1);
        dp[0] = 0;
        for (int i = 1; i <= amount; ++i) {
            for (int j = 0; j < coins.size(); ++j) {
                if (coins[j] <= i) {
                    dp[i] = min(dp[i], dp[i - coins[j]] + 1);
                }
            }
        }
        return dp[amount] > amount ? -1 : dp[amount];
    }
};

// Recursion
class Solution {
public:
    int coinChange(vector<int>& coins, int amount) {
        vector<int> dp(amount + 1, INT_MAX);
        dp[0] = 0;
        return coinChangeDFS(coins, amount, dp);
    }
    int coinChangeDFS(vector<int> &coins, int amount, vector<int> &dp) {
        if (amount < 0) return - 1;
        if (dp[amount] != INT_MAX) return dp[amount];
        for (int i = 0; i < coins.size(); ++i) {
            int tmp = coinChangeDFS(coins, amount - coins[i], dp);
            if (tmp >= 0) dp[amount] = min(dp[amount], tmp + 1);
        }
        return dp[amount] = dp[amount] == INT_MAX ? -1 : dp[amount];
    }
};
\end{lstlisting}


\section{Largest Divisible Subset (M)}
Given a set of distinct positive integers, find the largest subset such that every pair (Si, Sj) of elements in this subset satisfies: Si \% Sj = 0 or Sj \% Si = 0. If there are multiple solutions, return any subset is fine.\\

Example 1:
nums: [1,2,3]
Result: [1,2] (of course, [1,3] will also be ok)\\

Example 2:
nums: [1,2,4,8]
Result: [1,2,4,8]\\

\begin{lstlisting}
class Solution {
public:
    vector<int> largestDivisibleSubset(vector<int>& nums) {
        sort(nums.begin(), nums.end());
        vector<int> dp(nums.size(), 0), parent(nums.size(), 0), res;
        int mx = 0, mx_idx = 0;
        for (int i = nums.size() - 1; i >= 0; --i) {
            for (int j = i; j < nums.size(); ++j) {
                if (nums[j] % nums[i] == 0 && dp[i] < dp[j] + 1) {
                    dp[i] = dp[j] + 1;
                    parent[i] = j;
                    if (mx < dp[i]) {
                        mx = dp[i];
                        mx_idx = i;
                    }
                }
            }
        }
        for (int i = 0; i < mx; ++i) {
            res.push_back(nums[mx_idx]);
            mx_idx = parent[mx_idx];
        }
        return res;
    }
};
\end{lstlisting}


\section{Climbing Stairs (E)}
You are climbing a stair case. It takes n steps to reach to the top.\\

Each time you can either climb 1 or 2 steps. In how many distinct ways can you climb to the top? \\ 

\begin{lstlisting}
class Solution {
public:
    int climbStairs(int n) {
        vector<int> steps(n,0);
        steps[0] = 1;
        steps[1] = 2;
        // the number of distinct ways to reach level n is the sum of number of distinct ways to reach level n-1 and n-2.
        for(int i = 2; i < n; i++)
            steps[i] = steps[i-2] + steps[i-1]; 
        return steps[n-1];
    }
};
\end{lstlisting}


\section{Decode Ways (M)}
 A message containing letters from A-Z is being encoded to numbers using the following mapping:\\
'A' $->$ 1\\
'B' $->$ 2\\
...\\
'Z' $->$ 26\\

Given an encoded message containing digits, determine the total number of ways to decode it.\\

For example,
Given encoded message "12", it could be decoded as "AB" (1 2) or "L" (12). The number of ways decoding "12" is 2. \\

\begin{lstlisting}
class Solution(object):
    def numDecodings(self, s):
        """
        :type s: str
        :rtype: int
        """
        if not s or s[0] == '0': return 0
        dp = [0 for i in range(len(s) + 1)]
        dp[0] = 1
        # number of ways to decode until the digit i
        for i in range(1, len(dp)):
            # s[i-1] and s[i] cannot form a digit
            if s[i-1] != '0':
                dp[i] += dp[i-1]
            # s[i-2] and s[i-1] can form a digit in range [10, 26]
            if i >= 2 and int(s[i-2:i]) >= 10 and int(s[i-2:i]) <= 26:
                dp[i] += dp[i-2]
        return dp[-1]
 \end{lstlisting}

\begin{lstlisting}
class Solution {
public:
    int numDecodings(string s) {
        if (s.empty()) return 0;
        int n = s.size();
        vector<int> dp(n + 2, 1);
        for (int i = 2; i < n + 2; ++i) {
            if (s[i - 2] == '0') dp[i] = 0;
            else dp[i] = dp[i - 1];
            if (i - 3 >= 0 && (s[i - 3] == '1' || (s[i - 3] == '2' && s[i - 2] <= '6'))) {
                dp[i] += dp[i - 2];
            }
        }
        return dp[n + 1];
    }
};

class Solution {
public:
    int numDecodings(string s) {
        if (s.empty()) return 0;
        vector<int> dp(s.size() + 1, 0);
        dp[0] = 1;
        for (int i = 1; i < dp.size(); ++i) {
            if (s[i - 1] != '0') dp[i] += dp[i - 1];
            if (i >= 2 && s.substr(i - 2, 2) <= "26" && s.substr(i - 2, 2) >= "10") {
                dp[i] += dp[i - 2];
            }
        }
        return dp.back();
    }
};

// Space O(1)
class Solution {
public:
    int numDecodings(string s) {
        if (s.empty() || s.front() == '0') return 0;
        int c1 = 1, c2 = 1;
        for (int i = 1; i < s.size(); ++i) {
            if (s[i] == '0') c1 = 0;
            if (s[i - 1] == '1' || (s[i - 1] == '2' && s[i] <= '6')) {
                c1 = c1 + c2;
                c2 = c1 - c2;
            } else {
                c2 = c1;
            }
        }
        return c1;
    }
};
\end{lstlisting}

\section{Number of ways for mapping integers to alphabets (The Climate Corporation phone screen 2016.9.22)}
Define a mapping of integers to alphabets as follows: 1 = a, 2 = b, ..., 26 = z. Given any combination of of the mapping numbers as a string, return the number of ways in which the input string can be split into sub-strings and represented as character strings. \\

For example:\\
"111" $->$ "aaa", "ak", "ka" $->$ 3 ways\\
"11" $->$ "aa", "k" $->$ 2 ways\\
"123" $->$ "abc", "lc", "aw" $->$ 3 ways\\

This question is the same as "Decode Ways".\\

\begin{lstlisting}
#include <iostream>
#include <string>
#include <vector>
using namespace std;

int numOfWays(string);

int main() {
  string input = "123321";
  int res = numOfWays(input);
  cout << "String: " << input << "\n";
  cout << "Number of Combinations are: " << res << "\n";
  return 0;
}

int numOfWays(string s) {
  int n = s.size();
  vector<int> res(n);
  res[0] = 1;
  int tmp = (s[0] - '0') * 10 + (s[1] - '0');
  if (tmp <= 26) res[1] = 2;
  else res[1] = 1;
  
  for (int i = 2; i < n; ++i) {
    int tmp = (s[i-1] - '0') * 10 + (s[i] - '0');    
    if (tmp <= 26) res[i] = res[i-2] + res[i-1];
    else res[i] = res[i-1];
  }
    
  return res[n-1];
}
\end{lstlisting}


\section{Jump Game (M)}
Given an array of non-negative integers, you are initially positioned at the first index of the array. Each element in the array represents your maximum jump length at that position. Determine if you are able to reach the last index.\\

For example:\\
A = [2,3,1,1,4], return true.\\
A = [3,2,1,0,4], return false. \\

\begin{lstlisting}
class Solution(object):
    def canJump(self, nums):
        """
        :type nums: List[int]
        :rtype: bool
        """
        max_reach = 0
        n = len(nums)
        for i in range(n):
            if i > max_reach or max_reach >= n - 1:
                break
            # i + nums[i] is the maximum step that can be reached at i
            max_reach = max(max_reach, i + nums[i])
        return max_reach >= n - 1
\end{lstlisting}

\begin{lstlisting}
// 1. DP solution
class Solution {
public:
    bool canJump(vector<int>& nums) {
        int n = nums.size();
        // dp[i] is the maximum step that is left to jump at i
        vector<int> dp(n, 0);
        for (int i = 1; i < n; ++i) {
            dp[i] = max(dp[i - 1], nums[i - 1]) - 1;
            if (dp[i] < 0) return false;
        }
        return dp[n - 1] >= 0;
    }
};

// 2. Greedy solution
class Solution {
public:
    bool canJump(vector<int>& nums) {
        int maxIdx = 0, n = nums.size();
        for (int i = 0; i < n; ++i) {
            if (i > maxIdx || maxIdx >= n - 1) break;
            maxIdx = max(maxIdx, i + nums[i]);
        }
        return maxIdx >= n - 1;
    }
};
\end{lstlisting}


\section{Jump Game II (M)}
Given an array of non-negative integers, you are initially positioned at the first index of the array. Each element in the array represents your maximum jump length at that position. Your goal is to reach the last index in the minimum number of jumps.\\

For example:
Given array A = [2,3,1,1,4],
The minimum number of jumps to reach the last index is 2. (Jump 1 step from index 0 to 1, then 3 steps to the last index.)\\

Note:
You can assume that you can always reach the last index.\\

\begin{lstlisting}
/* Greedy solution
 * We use "last" to keep track of the maximum distance that has been reached
 * by using the minimum steps "res", whereas "curr" is the maximum distance
 * that can be reached by using "res+1" steps. Thus,
 * curr = max(i+nums[i]) where 0 <= i <= last.
 */
class Solution {
public:
    int jump(vector<int>& nums) {
        int res = 0, last = 0, cur = 0, n = nums.size();
        for (int i = 0; i < n; ++i) {
            if (i > last) {
                last = cur;
                ++res;
            }
            cur = max(cur, i + nums[i]);
        }
        return res;
    }
};
\end{lstlisting}


\section{Best Time to Buy and Sell Stock (E)}
Say you have an array for which the i-th element is the price of a given stock on day i. If you were only permitted to complete at most one transaction (ie, buy one and sell one share of the stock), design an algorithm to find the maximum profit.\\

\begin{lstlisting}
class Solution(object):
    def maxProfit(self, prices):
        """
        :type prices: List[int]
        :rtype: int
        """
        buy = float('inf')
        profit = 0
        for price in prices:
            if price < buy:
                buy = price
            # there is no need to check profit if we buy a stock
            # so we use elif here
            elif profit < price - buy:
                profit = price - buy
        return profit
\end{lstlisting}

\begin{lstlisting}
class Solution {
public:
    int maxProfit(vector<int>& prices) {
        if (prices.empty() || prices.size() < 2)
            return 0;
        
        int profit = 0;
        int low = prices[0];
        
        for (int i = 1; i < prices.size(); ++i) {
            profit = max(profit, prices[i] - low);
            low = min(low, prices[i]);
        }
        
        return profit;
    }
};
\end{lstlisting}


\section{Best Time to Buy and Sell Stock II (E)}
Say you have an array for which the i-th element is the price of a given stock on day i. \\

Design an algorithm to find the maximum profit. You may complete as many transactions as you like (ie, buy one and sell one share of the stock multiple times). However, you may not engage in multiple transactions at the same time (ie, you must sell the stock before you buy again). \\

\begin{lstlisting}
class Solution(object):
    def maxProfit(self, prices):
        """
        :type prices: List[int]
        :rtype: int
        """
        profit = 0
        for i in range(1, len(prices)):
            if prices[i] > prices[i-1]:
                profit += prices[i] - prices[i-1]
        return profit
\end{lstlisting}

\begin{lstlisting}
class Solution {
public:
    int maxProfit(vector<int>& prices) {
        if (prices.empty() || prices.size() < 2)
            return 0;
        
        int profit = 0;
        int diff;
        
        for (int i = 1; i < prices.size(); ++i) {
            diff = prices[i] - prices[i-1];
            if (diff > 0)
                profit += diff;
        }
        return profit;
    }
};
\end{lstlisting}


\section{Best Time to Buy and Sell Stock with Cooldown (M)}
Say you have an array for which the i-th element is the price of a given stock on day i.\\

Design an algorithm to find the maximum profit. You may complete as many transactions as you like (ie, buy one and sell one share of the stock multiple times) with the following restrictions:\\

    You may not engage in multiple transactions at the same time (ie, you must sell the stock before you buy again).\\
    After you sell your stock, you cannot buy stock on next day. (ie, cooldown 1 day)\\

\begin{lstlisting}
class Solution {
public:
    int maxProfit(vector<int>& prices) {

        int buy = INT_MIN, sell = 0, rest = INT_MIN, cooldown = 0;
        
        for (int i = 0; i < prices.size(); ++i) {
            rest = max(rest, buy);
            buy = cooldown - prices[i];
            cooldown = max(sell, cooldown);
            sell = rest + prices[i];
        }
        
        return max(cooldown, sell);
    }
};
\end{lstlisting}

\section{Best Time to Buy and Sell Stock with Transaction Fee (M)}
Your are given an array of integers prices, for which the i-th element is the price of a given stock on day i; and a non-negative integer fee representing a transaction fee.\\

You may complete as many transactions as you like, but you need to pay the transaction fee for each transaction. You may not buy more than 1 share of a stock at a time (ie. you must sell the stock share before you buy again.)\\

Return the maximum profit you can make.\\

\begin{lstlisting}
# class Solution(object):
#     def maxProfit(self, prices, fee):
#         """
#         :type prices: List[int]
#         :type fee: int
#         :rtype: int
#         """
#         n = len(prices)
#         sold = hold = [0] * n
#         hold[0] = -prices[0]
#         for i in range(1, n):
#             sold[i] = max(sold[i-1], hold[i-1] + prices[i] - fee)
#             hold[i] = max(hold[i-1], sold[i-1] - prices[i])
#         return sold[n-1]
    
class Solution(object):
    def maxProfit(self, prices, fee):
        cash, hold = 0, -prices[0]
        for i in range(1, len(prices)):
            cash = max(cash, hold + prices[i] - fee)
            hold = max(hold, cash - prices[i])
        return cash
\end{lstlisting}

\section{Best Time to Buy and Sell Stock III (H)}
Say you have an array for which the i-th element is the price of a given stock on day i. \\

Design an algorithm to find the maximum profit. You may complete at most two transactions. However, you may not engage in multiple transactions at the same time (ie, you must sell the stock before you buy again). \\

\begin{lstlisting}
class Solution {
public:
    int maxProfit(vector<int>& prices) {
        if (prices.empty() || prices.size() < 2)
            return 0;
        
        vector<int> profit(prices.size());
        
        // compute the forward max profit and save it
        int buy = prices[0];
        profit[0] = 0;
        for (int i = 1; i < prices.size(); i++) {
            profit[i] = max(profit[i - 1], prices[i] - buy);
            buy = min(buy, prices[i]);
        }
        
        // The final max profit is the sum of max profit before day i (profit[i]) and after day i (sell - prices[i])
        int sell = prices[prices.size() - 1];
        int best = 0;
        for (int i = prices.size() - 2; i >= 0; i--) {
            best = max(best, sell - prices[i] + profit[i]);
            sell = max(sell, prices[i]);
        }
        
        return best;   
    }
};
\end{lstlisting}


\section{Best Time to Buy and Sell Stock IV (H)}
Say you have an array for which the i-th element is the price of a given stock on day i. \\

Design an algorithm to find the maximum profit. You may complete at most k transactions. However, you may not engage in multiple transactions at the same time (ie, you must sell the stock before you buy again). \\

\begin{lstlisting}
class Solution {
public:
    int maxProfit(int k, vector<int> &prices) {
        if (prices.empty() || prices.size() < 2) 
            return 0;
        if (k >= prices.size()) 
            return solveMaxProfit(prices);
        
        int global[k + 1] = {0};
        int local[k + 1] = {0};
        for (int i = 0; i < prices.size() - 1; ++i) {
            int diff = prices[i + 1] - prices[i];
            for (int j = k; j >= 1; --j) {
                local[j] = max(global[j - 1] + max(diff, 0), local[j] + diff);
                global[j] = max(global[j], local[j]);
            }
        }
        return global[k];
    }
    
    int solveMaxProfit(vector<int> &prices) {
        int profit = 0;
        for (int i = 1; i < prices.size(); ++i) {
            if (prices[i] > prices[i - 1]) {
                profit += prices[i] - prices[i - 1];
            }
        }
        return profit;
    }
};
\end{lstlisting}


\section{House Robber (E)}
You are a professional robber planning to rob houses along a street. Each house has a certain amount of money stashed, the only constraint stopping you from robbing each of them is that adjacent houses have security system connected and it will automatically contact the police if two adjacent houses were broken into on the same night. \\

Given a list of non-negative integers representing the amount of money of each house, determine the maximum amount of money you can rob tonight without alerting the police. \\

\begin{lstlisting}
class Solution(object):
    def rob(self, nums):
        """
        :type nums: List[int]
        :rtype: int
        """
        rob, not_rob = 0, 0
        n = len(nums)
        for i in range(n):
            pre_rob, pre_not_rob = rob, not_rob
            rob = pre_not_rob + nums[i]
            not_rob = max(pre_rob, pre_not_rob)
        return max(rob, not_rob)
\end{lstlisting}

\begin{lstlisting}
class Solution {
public:
    int rob(vector<int>& nums) {
        int cur_rob = 0, prev_rob = 0, sum = 0;
        
        for (int i = 0; i < nums.size(); ++i) {
            cur_rob = prev_rob + nums[i];
            prev_rob = sum;
            sum = max(cur_rob, prev_rob);
        }
        
        return sum;
    }
};
\end{lstlisting}


\section{House Robber II (M)}
After robbing those houses on that street, the thief has found himself a new place for his thievery so that he will not get too much attention. This time, all houses at this place are arranged in a circle. That means the first house is the neighbor of the last one. Meanwhile, the security system for these houses remain the same as for those in the previous street. \\

Given a list of non-negative integers representing the amount of money of each house, determine the maximum amount of money you can rob tonight without alerting the police.\\

\begin{lstlisting}
class Solution {
public:
    int rob(vector<int>& nums) {
        int n = nums.size();
        if (n == 0) return 0;
        if (n == 1) return nums[0];
        return max(rob(nums, 0, n-2), rob(nums, 1, n-1));     // can not rob nums[0] and nums[n-1] together
    }
    
    int rob(vector<int> &nums, int start, int end) {
        int cur_rob = 0, prev_rob = 0, sum = 0;
        
        for (int i = start; i <= end; ++i) {
            cur_rob = prev_rob + nums[i];
            prev_rob = sum;
            sum = max(cur_rob, prev_rob);
        }
        
        return sum;
    }
};
\end{lstlisting}


\section{House Robber III (M)}
The thief has found himself a new place for his thievery again. There is only one entrance to this area, called the "root." Besides the root, each house has one and only one parent house. After a tour, the smart thief realized that "all houses in this place forms a binary tree". It will automatically contact the police if two directly-linked houses were broken into on the same night. \\

Determine the maximum amount of money the thief can rob tonight without alerting the police. \\

\begin{lstlisting}
class Solution {
public:
    int rob(TreeNode *root) {
        vector<int> res = robber(root);
        return max(res[0], res[1]);
    }
    
    vector<int> robber(TreeNode *root) {
        vector<int> res(2,0);
        if (!root)  return res;
        
        vector<int> left = robber(root->left);
        vector<int> right = robber(root->right);
        
        res[0] = max(left[0], left[1]) + max(right[0], right[1]);   // if root is not robbed
        res[1] = root->val + left[0] + right[0];                    // if root is robbed
        
        return res;
    }
};
\end{lstlisting}


\section{Paint Fence (E)}
There is a fence with n posts, each post can be painted with one of the k colors. You have to paint all the posts such that no more than two adjacent fence posts have the same color. Return the total number of ways you can paint the fence.\\

Note: n and k are non-negative integers.\\

\begin{lstlisting}
class Solution {
public:
    int numWays(int n, int k) {
        if (n == 0) return 0;
        if (n == 1) return k;
        vector<int> dp(n);
        
        dp[0] = k;                          // n = 1, k ways to paint
        dp[1] = k * (k - 1) + k;            // n = 2, diff color: k*(k-1) ways + same color: k ways
        
        /** 1. If the color of the current post i is different from the color of the last post i-1,
         *      then there are dp[i] = dp[i - 1] * (k - 1) ways to paint the current post i
         *  2. If the color of the current post i is same as the color of the last post i-1,
         *      then the color of the post i and i-1 must be different from the color of the second last post i-2
         *      so there are dp[i] = dp[i - 2] * (k - 1) * 1 ways to paint the current post i  
         *  3. The total num of ways is a combination of case 1 and 2
         */
        for (int i = 2; i < n; i++) {
            dp[i] = dp[i - 1] * (k - 1) + dp[i - 2] * (k - 1);
        }
        
        return dp[n - 1];
    }
};
\end{lstlisting}


\section{Paint House (M)}
There are a row of n houses, each house can be painted with one of the three colors: red, blue or green. The cost of painting each house with a certain color is different. You have to paint all the houses such that no two adjacent houses have the same color.\\

The cost of painting each house with a certain color is represented by a n x 3 cost matrix. For example, costs[0][0] is the cost of painting house 0 with color red; costs[1][2] is the cost of painting house 1 with color green, and so on... Find the minimum cost to paint all houses.\\

Note:
All costs are positive integers.\\

\begin{lstlisting}
class Solution {
public:
    int minCost(vector<vector<int>> &cost) {
        if (cost.empty() || cost[0].empty())    return 0;
        vector<vector<int>> dp = costs;
        for (int i = 1; i < dp.size(); ++i) {
            // dp[i][0] += min(dp[i - 1][1], dp[i - 1][2]);
            // dp[i][1] += min(dp[i - 1][0], dp[i - 1][2]);
            // dp[i][2] += min(dp[i - 1][0], dp[i - 1][1]);
            for (int j = 0; j < 3; ++j) {
                dp[i][j] += min(dp[i-1][(j+1)%3], dp[i-1][(j+2)%3]);
            }
        }
        return min(min(dp.back()[0], dp.back()[1]), dp.back()[2]);
    }
};
\end{lstlisting}


\section{Paint House II (H)}
There are a row of n houses, each house can be painted with one of the k colors. The cost of painting each house with a certain color is different. You have to paint all the houses such that no two adjacent houses have the same color.\\

The cost of painting each house with a certain color is represented by a n x k cost matrix. For example, costs[0][0] is the cost of painting house 0 with color 0; costs[1][2]is the cost of painting house 1 with color 2, and so on... Find the minimum cost to paint all houses.\\

Note:
All costs are positive integers.\\

Follow up:
Could you solve it in O(nk) runtime?\\

\begin{lstlisting}
class Solution {
public:
    int minCostII(vector<vector<int>>& costs) {
        if(costs.empty() || costs[0].empty())   return 0;
        int n = costs.size(), k = costs[0].size(), res = INT_MAX;
        vector<vector<int>> dp = costs;
        for (int i = 1; i < n; ++i) {
            for (int j = 0; j < k; ++j) {
                int tmp = INT_MAX;
                // find the local min cost of using other color to paint the last house
                for (int d = 1; d < k; ++d) {
                    tmp = min(tmp, dp[i-1][(j+d)%k]);
                }
                dp[i][j] += tmp;
                // find the global min cost of painting all houses
                if (i == n-1) {
                    res = min(res, dp[i][j]);
                }
            }
        }
        return res;
    }
};

class Solution {
public:
    int minCostII(vector<vector<int>>& costs) {
        if (costs.empty() || costs[0].empty()) return 0;
        vector<vector<int>> dp = costs;
        int min1 = -1, min2 = -1;
        for (int i = 0; i < dp.size(); ++i) {
            int last1 = min1, last2 = min2;
            min1 = -1; min2 = -1;
            for (int j = 0; j < dp[i].size(); ++j) {
                if (j != last1) {
                    dp[i][j] += last1 < 0 ? 0 : dp[i - 1][last1];
                } else {
                    dp[i][j] += last2 < 0 ? 0 : dp[i - 1][last2];
                }
                if (min1 < 0 || dp[i][j] < dp[i][min1]) {
                    min2 = min1; min1 = j;
                } else if (min2 < 0 || dp[i][j] < dp[i][min2]) {
                    min2 = j;
                }
            }
        }
        return dp.back()[min1];
    }
};
\end{lstlisting}


\section{Pascal's Triangle (E)}
Given numRows, generate the first numRows of Pascal's triangle. \\

\begin{lstlisting}
class Solution {
public:
    vector<vector<int>> generate(int numRows) {
        vector<vector<int>> res;
        if (numRows == 0)   return res;
        
        res.push_back(vector<int> (1,1));               // first row
        
        for (int i = 2; i <= numRows; ++i) {
            vector<int> cur(i,1);                       // generate next row with all 1s
            for (int j = 1; j < i-1; ++j)               // update elements from the 2nd to the (i-1)-th 
                cur[j] = res[i-2][j-1] + res[i-2][j];
            res.push_back(cur);
        }
        
        return res;
    }
};
\end{lstlisting}


\section{Pascal's Triangle II (E)}
Given an index k, return the k-th row of the Pascal's triangle. \\

\begin{lstlisting}
class Solution {
public:
    vector<int> getRow(int rowIndex) {
        vector<int> res;
        
        for (int i = 0; i <= rowIndex; ++i) {
            for (int j = i - 1; j > 0; --j) {
                res[j] = res[j-1] + res[j];     // scrolling array
            }
            res.push_back(1);
        }
        
        return res;
    }
};
\end{lstlisting}


\section{Range Sum Query - Immutable (E)}
Given an integer array nums, find the sum of the elements between indices i and j (i $\leq$ j), inclusive.\\

Example: \\
Given nums = [-2, 0, 3, -5, 2, -1]\\
sumRange(0, 2) = 1\\
sumRange(2, 5) = -1\\
sumRange(0, 5) = -3\\

Note:\\
    You may assume that the array does not change.\\
    There are many calls to sumRange function.\\

\begin{lstlisting}
class NumArray(object):

    def __init__(self, nums):
        """
        :type nums: List[int]
        """
        self.dp = [0 for i in range(len(nums)+1)]
        for i in range(len(nums)):
            self.dp[i+1] = self.dp[i] + nums[i]

    def sumRange(self, i, j):
        """
        :type i: int
        :type j: int
        :rtype: int
        """
        return self.dp[j+1] - self.dp[i]


# Your NumArray object will be instantiated and called as such:
# obj = NumArray(nums)
# param_1 = obj.sumRange(i,j)
\end{lstlisting}


\begin{lstlisting}
// Your NumArray object will be instantiated and called as such:
// NumArray numArray(nums);
// numArray.sumRange(0, 1);
// numArray.sumRange(1, 2);

class NumArray {
public:
    vector<int> sums = {0};             // save an initial 0 into sums
    
    NumArray(vector<int> &nums) {       // class constructor
        int sum = 0;
        for (int i = 0; i < nums.size(); ++i) {
            sum += nums[i];
            sums.push_back(sum);        // sums[] contains nums.size() + 1 elements
        }
    }
    
    int sumRange(int i, int j) {
        return sums[j+1] - sums[i];     // get the correct sum by considering the offset in sums[]
    }
};
\end{lstlisting}


\section{Range Sum Query - Mutable (M)}
Given an integer array nums, find the sum of the elements between indices i and j (i $\leq$ j), inclusive. The update(i, val) function modifies nums by updating the element at index i to val. \\

Example: \\
Given nums = [1, 3, 5]
sumRange(0, 2) = 9
update(1, 2)
sumRange(0, 2) = 8

Note:\\
    The array is only modifiable by the update function.\\
    You may assume the number of calls to update and sumRange function is distributed evenly.\\

\begin{lstlisting}
struct SegmentTreeNode {
    int start, end, sum;
    SegmentTreeNode* left;
    SegmentTreeNode* right;
    SegmentTreeNode(int a, int b):start(a),end(b),sum(0),left(nullptr),right(nullptr){}
};
class NumArray {
    SegmentTreeNode* root;
public:
    NumArray(vector<int> &nums) {
        int n = nums.size();
        root = buildTree(nums,0,n-1);
    }
   
    void update(int i, int val) {
        modifyTree(i,val,root);
    }

    int sumRange(int i, int j) {
        return queryTree(i, j, root);
    }
    SegmentTreeNode* buildTree(vector<int> &nums, int start, int end) {
        if(start > end) return nullptr;
        SegmentTreeNode* root = new SegmentTreeNode(start,end);
        if(start == end) {
            root->sum = nums[start];
            return root;
        }
        int mid = start + (end - start) / 2;
        root->left = buildTree(nums,start,mid);
        root->right = buildTree(nums,mid+1,end);
        root->sum = root->left->sum + root->right->sum;
        return root;
    }
    int modifyTree(int i, int val, SegmentTreeNode* root) {
        if(root == nullptr) return 0;
        int diff;
        if(root->start == i && root->end == i) {
            diff = val - root->sum;
            root->sum = val;
            return diff;
        }
        int mid = (root->start + root->end) / 2;
        if(i > mid) {
            diff = modifyTree(i,val,root->right);
        } else {
            diff = modifyTree(i,val,root->left);
        }
        root->sum = root->sum + diff;
        return diff;
    }
    int queryTree(int i, int j, SegmentTreeNode* root) {
        if(root == nullptr) return 0;
        if(root->start == i && root->end == j) return root->sum;
        int mid = (root->start + root->end) / 2;
        if(i > mid) return queryTree(i,j,root->right);
        if(j <= mid) return queryTree(i,j,root->left);
        return queryTree(i,mid,root->left) + queryTree(mid+1,j,root->right);
    }
};


// Your NumArray object will be instantiated and called as such:
// NumArray numArray(nums);
// numArray.sumRange(0, 1);
// numArray.update(1, 10);
// numArray.sumRange(1, 2);
\end{lstlisting}


\section{Range Sum Query 2D - Immutable (M)}
Given a 2D matrix matrix, find the sum of the elements inside the rectangle defined by its upper left corner (row1, col1) and lower right corner (row2, col2). \\

Example:\\
Given matrix = [
  [3, 0, 1, 4, 2], 
  [5, 6, 3, 2, 1], 
  [1, 2, 0, 1, 5], 
  [4, 1, 0, 1, 7], 
  [1, 0, 3, 0, 5] 
]\\
sumRegion(2, 1, 4, 3) = 8\\
sumRegion(1, 1, 2, 2) = 11\\
sumRegion(1, 2, 2, 4) = 12\

Note:\\
    You may assume that the matrix does not change.\\
    There are many calls to sumRegion function.\\
    You may assume that row1 $\leq$ row2 and col1 $\leq$ col2.\\

\begin{lstlisting}
class NumMatrix(object):

    def __init__(self, matrix):
        """
        :type matrix: List[List[int]]
        """
        if len(matrix) == 0 or len(matrix[0]) == 0: return
        m, n = len(matrix), len(matrix[0])
        self.dp = [ [0 for j in range(n+1)] for i in range(m+1) ]
        # note that the index in dp is 1 larger than the index in matrix
        for row in range(m):
            for col in range(n):
                self.dp[row+1][col+1] = self.dp[row+1][col] + self.dp[row][col+1] \
                                        + matrix[row][col] - self.dp[row][col]

    def sumRegion(self, row1, col1, row2, col2):
        """
        :type row1: int
        :type col1: int
        :type row2: int
        :type col2: int
        :rtype: int
        """
        return self.dp[row2 + 1][col2 + 1] - self.dp[row1][col2 + 1] \
                - self.dp[row2 + 1][col1] + self.dp[row1][col1]       


# Your NumMatrix object will be instantiated and called as such:
# obj = NumMatrix(matrix)
# param_1 = obj.sumRegion(row1,col1,row2,col2)
\end{lstlisting}


\begin{lstlisting}
class NumMatrix {
public:
    int row, col;
    vector<vector<int>> sums;
    
    NumMatrix(vector<vector<int>> &matrix) {
        row = matrix.size();
        col = row > 0 ? matrix[0].size() : 0;
        sums = vector<vector<int>>(row+1, vector<int>(col+1, 0));
        for(int i = 1; i <= row; i++) {
            for(int j = 1; j <= col; j++) {
                sums[i][j] = sums[i-1][j] + sums[i][j-1] - sums[i-1][j-1] + matrix[i-1][j-1];
            }
        }
    }

    int sumRegion(int row1, int col1, int row2, int col2) {
        return sums[row2+1][col2+1] - sums[row2+1][col1] - sums[row1][col2+1] + sums[row1][col1];
    }
};

// Your NumMatrix object will be instantiated and called as such:
// NumMatrix numMatrix(matrix);
// numMatrix.sumRegion(0, 1, 2, 3);
// numMatrix.sumRegion(1, 2, 3, 4);
\end{lstlisting}


\section{Triangle (M)}
Given a triangle, find the minimum path sum from top to bottom. Each step you may move to adjacent numbers on the row below.\\

For example, given the following triangle
[
     [2],
    [3,4],
   [6,5,7],
  [4,1,8,3]
]
The minimum path sum from top to bottom is 11 (i.e., 2 + 3 + 5 + 1 = 11).\\

Note:
Bonus point if you are able to do this using only O(n) extra space, where n is the total number of rows in the triangle. \\

\begin{lstlisting}
// bottom-up approach
class Solution {
public:
    int minimumTotal(vector<vector<int>>& triangle) {
        vector<int> res = triangle.back();
        for (int i = triangle.size() - 2; i >= 0; --i) {
            for (int j = 0; j <= i; ++j) {
                res[j] = min(res[j], res[j+1]) + triangle[i][j];
            }
        }
        return res[0];
    }
};
\end{lstlisting}


\section{Unique Paths (M)}
A robot is located at the top-left corner of a m x n grid (marked 'Start' in the diagram below). The robot can only move either down or right at any point in time. The robot is trying to reach the bottom-right corner of the grid (marked 'Finish' in the diagram below). How many possible unique paths are there?\\

Note: m and n will be at most 100.\\

\begin{lstlisting}
// 1. 2D DP solution
class Solution {
public:
    int uniquePaths(int m, int n) {
        vector<vector<int>> dp(m, vector<int>(n,1));
        for (int i = 1; i < m; ++i) {
            for (int j = 1; j < n; ++j) {
                dp[i][j] = dp[i-1][j] + dp[i][j-1];
            }
        }
        return dp[m-1][n-1];
    }
};

// 2. 1D DP solution
class Solution {
public:
    int uniquePaths(int m, int n) {
        vector<int> dp(n,1);
        for (int i = 1; i < m; ++i) {
            for (int j = 1; j < n; ++j) {
                dp[j] = dp[j] + dp[j-1];
            }
        }
        return dp[n-1];
    }
};
\end{lstlisting}


\section{Unique Paths II (M)}
Follow up for "Unique Paths":

Now consider if some obstacles are added to the grids. How many unique paths would there be? An obstacle and empty space is marked as 1 and 0 respectively in the grid.\\

For example, There is one obstacle in the middle of a 3x3 grid as illustrated below.
[
  [0,0,0],
  [0,1,0],
  [0,0,0]
], The total number of unique paths is 2.\\

Note: m and n will be at most 100.\\

\begin{lstlisting}
// 1. 2D DP
class Solution {
public:
    int uniquePathsWithObstacles(vector<vector<int>>& obstacleGrid) {
        if (obstacleGrid.empty() || obstacleGrid[0].empty()) return 0;
        int m = obstacleGrid.size(), n = obstacleGrid[0].size();
        if (obstacleGrid[0][0] == 1 || obstacleGrid[m-1][n-1] == 1) return 0;
        vector<vector<int>> dp(m, vector<int>(n, 0));
        for (int i = 0; i < m; ++i) {
            for (int j = 0; j < n; ++j) {
                if (obstacleGrid[i][j] == 1) {
                    dp[i][j] = 0;
                } else if (i == 0 && j == 0) {
                    dp[i][j] = 1;
                } else if (i == 0 && j > 0) {
                    dp[i][j] = dp[i][j-1];
                } else if (i > 0 && j == 0) {
                    dp[i][j] = dp[i-1][j];
                } else {
                    dp[i][j] = dp[i-1][j] + dp[i][j-1];
                }
            }
        }
        return dp[m-1][n-1];
    }
};

// 2. 1D DP
class Solution {
public:
    int uniquePathsWithObstacles(vector<vector<int>>& obstacleGrid) {
        if (obstacleGrid.empty() || obstacleGrid[0].empty()) return 0;
        int m = obstacleGrid.size(), n = obstacleGrid[0].size();
        if (obstacleGrid[0][0] == 1 || obstacleGrid[m-1][n-1] == 1) return 0;
        vector<int> dp(n, 0);
        dp[0] = 1;
        for (int i = 0; i < m; ++i) {
            for (int j = 0; j < n; ++j) {
                if (obstacleGrid[i][j] == 1) {
                    dp[j] = 0;
                } else if (j > 0) {
                    dp[j] = dp[j] + dp[j-1];
                }
            }
        }
        return dp[n-1];
    }
};
\end{lstlisting}


\section{Minimum Path Sum (M)}
Given a m x n grid filled with non-negative numbers, find a path from top left to bottom right which minimizes the sum of all numbers along its path.\\

Note: You can only move either down or right at any point in time.\\

\begin{lstlisting}
// 1. 2D DP
class Solution {
public:
    int minPathSum(vector<vector<int>>& grid) {
        if (grid.size() == 0 || grid[0].size() == 0) return 0;
        int m = grid.size(), n = grid[0].size();
        int dp[m][n];
        // initializing
        dp[0][0] = grid[0][0];
        for (int i = 1; i < m; ++i) dp[i][0] = dp[i-1][0] + grid[i][0];
        for (int j = 1; j < n; ++j) dp[0][j] = dp[0][j-1] + grid[0][j];
        // get min
        for (int i = 1; i < m; ++i) {
            for (int j = 1; j < n; ++j) {
                dp[i][j] = grid[i][j] + min(dp[i-1][j], dp[i][j-1]);
            }
        }    
        return dp[m-1][n-1];
    }
};

// 2. 1D DP
class Solution {
public:
    int minPathSum(vector<vector<int>>& grid) {
        if (grid.size() == 0 || grid[0].size() == 0) return 0;
        int m = grid.size(), n = grid[0].size();
        int dp[n];
        dp[0] = grid[0][0];
        for (int i = 1; i < n; ++i) dp[i] = dp[i-1] + grid[0][i];
        for (int i = 1; i < m; ++i) {
            dp[0] += grid[i][0]; // need to update dp[0] for each row
            for (int j = 1; j < n; ++j) {
                dp[j] = grid[i][j] + min(dp[j-1], dp[j]);
            }
        }
        return dp[n-1];
    }
};
\end{lstlisting}


\section{Dungeon Game (H)}
The demons had captured the princess (P) and imprisoned her in the bottom-right corner of a dungeon. The dungeon consists of M x N rooms laid out in a 2D grid. Our valiant knight (K) was initially positioned in the top-left room and must fight his way through the dungeon to rescue the princess. \\

The knight has an initial health point represented by a positive integer. If at any point his health point drops to 0 or below, he dies immediately. Some of the rooms are guarded by demons, so the knight loses health (negative integers) upon entering these rooms; other rooms are either empty (0's) or contain magic orbs that increase the knight's health (positive integers).\\

In order to reach the princess as quickly as possible, the knight decides to move only rightward or downward in each step. Write a function to determine the knight's minimum initial health so that he is able to rescue the princess.\\

Notes:\\
    The knight's health has no upper bound.\\
    Any room can contain threats or power-ups, even the first room the knight enters and the bottom-right room where the princess is imprisoned.\\

\begin{lstlisting}
class Solution {
public:
    int calculateMinimumHP(vector<vector<int>>& dungeon) {
        int m = dungeon.size(), n = dungeon[0].size();
        int dp[m][n];
        // Initialize K's health as K has to be alive when K reaches P
        // 1. If the current room can increase K's HP, 
        // then the minimum HP for K to reach the room is 1
        // 2. If the current room can reduce K's HP,
        // then the minimum HP for K to reach the room must be 1 - damage
        dp[m-1][n-1] = max(1, 1 - dungeon[m-1][n-1]);
        // initializing the last column
        for (int i = m - 2; i >= 0; --i) {
            dp[i][n-1] = max(1, dp[i+1][n-1] - dungeon[i][n-1]); 
        }
        // initializing the last row
        for (int j = n - 2; j >= 0; --j) {
            dp[m-1][j] = max(1, dp[m-1][j+1] - dungeon[m-1][j]); 
        }
        for (int i = m - 2; i >= 0; --i) {
            for (int j = n - 2; j >= 0; --j) {
                dp[i][j] = max(1, min(dp[i+1][j], dp[i][j+1]) - dungeon[i][j]);
            }
        }
        return dp[0][0];
    }
};
\end{lstlisting}


\section{Increasing Triplet Subsequence (M)}
Given an unsorted array return whether an increasing subsequence of length 3 exists or not in the array.\\

Formally the function should:\\
    Return true if there exists i, j, k\\
    such that $arr[i] < arr[j] < arr[k]$ given $0 \leq i < j < k \leq n-1$ else return false. \\

Your algorithm should run in O(n) time complexity and O(1) space complexity.\\

Examples:\\
Given [1, 2, 3, 4, 5],
return true.\\
Given [5, 4, 3, 2, 1],
return false. \\

\begin{lstlisting}
# # DP: the same as Longest Increasing Subsequence
# class Solution(object):
#     def increasingTriplet(self, nums):
#         """
#         :type nums: List[int]
#         :rtype: bool
#         """
#         n = len(nums)
#         dp = [1] * n
#         for i in range(n):
#             for j in range(i):
#                 if nums[j] < nums[i]:
#                     dp[i] = max(dp[i], dp[j] + 1)
#                     if dp[i] == 3: return True
#         return False

# O(N)
class Solution(object):
    def increasingTriplet(self, nums):
        """
        :type nums: List[int]
        :rtype: bool
        """
        mx1 = mx2 = float('inf')
        for num in nums:
            if num <= mx1: mx1 = num
            elif num <= mx2: mx2 = num
            else: return True
        return False
\end{lstlisting}

\begin{lstlisting}
// 1. DP solution: Time O(N^2), Space O(N)
class Solution {
public:
    bool increasingTriplet(vector<int>& nums) {
        vector<int> dp(nums.size(), 1);
        for (int i = 0; i < nums.size(); ++i) {
            for (int j = 0; j < i; ++j) {
                if (nums[j] < nums[i]) {
                    dp[i] = max(dp[i], dp[j] + 1);
                    if (dp[i] == 3) return true;
                }
            }
        }
        return false;
    }
};

// 2. Two minimum values: Time O(N), Space O(1)
class Solution {
public:
    bool increasingTriplet(vector<int>& nums) {
        int c1 = INT_MAX, c2 = INT_MAX;
        for (int i = 0; i < nums.size(); ++i) {
            if (nums[i] <= c1) c1 = nums[i]; // the first minimum
            else if (nums[i] <= c2) c2 = nums[i]; // the second minimum
            else return true; // the third value
        }
        return false;
    }
};
\end{lstlisting}

\section{Longest Continuous Increasing Subsequence (E)}
 Given an unsorted array of integers, find the length of longest continuous increasing subsequence (subarray).\\

Example 1:\\

Input: [1,3,5,4,7]\\
Output: 3\\
Explanation: The longest continuous increasing subsequence is [1,3,5], its length is 3. \\
Even though [1,3,5,7] is also an increasing subsequence, it's not a continuous one where 5 and 7 are separated by 4. \\

Example 2:\\

Input: [2,2,2,2,2]\\
Output: 1\\
Explanation: The longest continuous increasing subsequence is [2], its length is 1. \\

Note: Length of the array will not exceed 10,000. \\
\begin{lstlisting}
class Solution(object):
    def findLengthOfLCIS(self, nums):
        """
        :type nums: List[int]
        :rtype: int
        """
        cnt, res = 0, 0
        for i in range(len(nums)):
            if i == 0 or nums[i-1] < nums[i]:
                cnt += 1
                res = max(res, cnt)
            else:
                cnt = 1 # reset cnt if the continuous increasing stops
        return res
\end{lstlisting}

\section{Longest Increasing Subsequence (M)}
Given an unsorted array of integers, find the length of longest increasing subsequence.\\

For example,
Given [10, 9, 2, 5, 3, 7, 101, 18],
The longest increasing subsequence is [2, 3, 7, 101], therefore the length is 4. Note that there may be more than one LIS combination, it is only necessary for you to return the length.\\

Your algorithm should run in $O(n^2)$ complexity.\\

Follow up: Could you improve it to O(n log n) time complexity? \\

\begin{lstlisting}
class Solution(object):
    def lengthOfLIS(self, nums):
        """
        :type nums: List[int]
        :rtype: int
        """
        n = len(nums)
        dp = [1] * n
        res = 0
        for i in range(n):
            for j in range(i):
                if nums[j] < nums[i]:
                    dp[i] = max(dp[i], dp[j] + 1)
            res = max(res, dp[i])
        return res
\end{lstlisting}

\begin{lstlisting}
// 1. DP solution: Time O(N^2), Space O(N)
class Solution {
public:
    int lengthOfLIS(vector<int>& nums) {
        vector<int> dp(nums.size(), 1);
        int res = 0;
        for (int i = 0; i < nums.size(); ++i) {
            for (int j = 0; j < i; ++j) {
                if (nums[j] < nums[i]) {
                    dp[i] = max(dp[i], dp[j] + 1);
                }
            }
            res = max(res, dp[i]);
        }
        return res;
    }
};

// 2. Binary search: Time O(NlogN), Space O(N)
class Solution {
public:
    int lengthOfLIS(vector<int>& nums) {
        if (nums.empty()) return 0;
        vector<int> res = {nums[0]};
        for (int i = 1; i < nums.size(); ++i) {
            if (nums[i] < res[0]) { // update minimum
                res[0] = nums[i];
            } else if (nums[i] > res.back()) { // update maximum
                res.push_back(nums[i]);
            } else { // find values between the first and the last element of res and add them into res
                int left = 0, right = res.size() - 1;
                while (left < right) {
                    int mid = left + (right - left) / 2;
                    if (res[mid] < nums[i]) left = mid + 1;
                    else right = mid;
                }
                res[right] = nums[i];
            }
        }
        return res.size(); // the res size is the length of LIS
    }
};
\end{lstlisting}


\section{Number of Longest Increasing Subsequence (M)}
 Given an unsorted array of integers, find the number of longest increasing subsequence.\\

Example 1:\\

Input: [1,3,5,4,7]\\
Output: 2\\
Explanation: The two longest increasing subsequence are [1, 3, 4, 7] and [1, 3, 5, 7].\\

Example 2:\\

Input: [2,2,2,2,2]\\
Output: 5\\
Explanation: The length of longest continuous increasing subsequence is 1, and there are 5 subsequences' length is 1, so output 5.\\

Note: Length of the given array will be not exceed 2000 and the answer is guaranteed to be fit in 32-bit signed int. \\
\begin{lstlisting}
class Solution(object):
    def findNumberOfLIS(self, nums):
        """
        :type nums: List[int]
        :rtype: int
        """
        mx, res = 0, 0
        n = len(nums)
        # length of the increasing subsequence (IS) at i
        length = [1] * n
        # count of the increasing subsequence at i 
        cnt = [1] * n
        for i in range(n):
            for j in range(i):
                # skip if not IS
                if nums[i] <= nums[j]: continue
                # nums[i] can be added after the IS that is end at nums[j]
                # cnt[j] can be added into cnt[i]
                if length[i] == length[j] + 1:
                    cnt[i] += cnt[j]
                # a longer IS exists at nums[i], so update length[i]
                if length[i] < length[j] + 1:
                    length[i] = length[j] + 1
                    cnt[i] = cnt[j]
            if mx == length[i]:
                res += cnt[i]
            elif mx < length[i]:
                mx = length[i]
                res = cnt[i]
        return res
\end{lstlisting}

\section{Longest Consecutive Sequence (H)}
Given an unsorted array of integers, find the length of the longest consecutive elements sequence.\\

Your algorithm should run in O(n) complexity.\\

Example:\\

Input: [100, 4, 200, 1, 3, 2]\\
Output: 4\\
Explanation: The longest consecutive elements sequence is [1, 2, 3, 4]. Therefore its length is 4.\\

\begin{lstlisting}
# # O(N^3)
# class Solution(object):
#     def longestConsecutive(self, nums):
#         """
#         :type nums: List[int]
#         :rtype: int
#         """
#         longest = 0
#         for num in nums:
#             cur_num = num
#             cur = 1
#             while cur_num + 1 in nums:
#                 cur_num += 1
#                 cur += 1
#             longest = max(longest, cur)
#         return longest
    
# O(NlogN)
class Solution(object):
    def longestConsecutive(self, nums):
        if not nums: return 0
        nums.sort()
        cur, longest = 1, 1
        for i in range(1, len(nums)):
            # skip duplicates
            if nums[i] != nums[i-1]:
                # find the consecutive sequence
                if nums[i] == nums[i-1] + 1:
                    cur += 1
                    longest = max(longest, cur)
                # otherwise update longest and reset cur
                else:
                    cur = 1
        return longest
           
# O(N)
class Solution(object):
    def longestConsecutive(self, nums):
        longest = 0
        num_set = set(nums)
        for num in num_set:
        # We only attempt to build sequences from numbers that are not already 
        # part of a longer sequence. This is accomplished by first ensuring that 
        # the number that would immediately precede the current number in a sequence 
        # is not present, as that number would necessarily be part of a longer sequence.
            if num - 1 not in num_set:
                cur_num = num
                cur = 1
                while cur_num + 1 in num_set:
                    cur_num += 1
                    cur +=1
                longest = max(longest, cur)
        return longest
\end{lstlisting}
        

\section{Longest Increasing Path in a Matrix (M)}
Given an integer matrix, find the length of the longest increasing path. From each cell, you can either move to four directions: left, right, up or down. You may NOT move diagonally or move outside of the boundary (i.e. wrap-around is not allowed).\\

Example 1:\\
nums = [
  [9,9,4],
  [6,6,8],
  [2,1,1]
],
Return 4,
The longest increasing path is [1, 2, 6, 9].\\

Example 2:\\
nums = [
  [3,4,5],
  [3,2,6],
  [2,2,1]
],
Return 4,
The longest increasing path is [3, 4, 5, 6]. Moving diagonally is not allowed.\\

\begin{lstlisting}
class Solution {
public:
    int longestIncreasingPath(vector<vector<int> >& matrix) {
        if (matrix.empty() || matrix[0].empty()) return 0;
        int res = 1, m = matrix.size(), n = matrix[0].size();
        vector<vector<int> > dp(m, vector<int>(n, 0));
        for (int i = 0; i < m; ++i) {
            for (int j = 0; j < n; ++j) {
                res = max(res, dfs(matrix, dp, i, j));
            }
        }
        return res;
    }
    int dfs(vector<vector<int> > &matrix, vector<vector<int> > &dp, int i, int j) {
        if (dp[i][j]) return dp[i][j];
        // move left, up, right, down
        vector<vector<int> > dirs = {{0, -1}, {-1, 0}, {0, 1}, {1, 0}}; 
        int max_len = 1, m = matrix.size(), n = matrix[0].size();
        for (auto a : dirs) {
            int x = i + a[0], y = j + a[1]; // move (i,j) to next cell
            // avoid corner cases, look for the increasing path
            if (x < 0 || x >= m || y < 0 || y >= n || matrix[x][y] <= matrix[i][j]) continue;
            int len = 1 + dfs(matrix, dp, x, y);
            max_len = max(max_len, len);
        }
        dp[i][j] = max_len;
        return max_len;
    }
};
\end{lstlisting}


\section{Maximal Square (M)}
Given a 2D binary matrix filled with 0's and 1's, find the largest square containing all 1's and return its area.\\

For example, given the following matrix:\\
1 0 1 0 0\\
1 0 1 1 1\\
1 1 1 1 1\\
1 0 0 1 0\\
Return 4.\\

\begin{lstlisting}
class Solution {
public:
    int maximalSquare(vector<vector<char>>& matrix) {
        if(matrix.size() == 0) return 0;
        int maxSq = 0;
        int nRow = matrix.size();
        int nCol = matrix[0].size();
        vector<vector<int>> dp(nRow + 1, vector<int>(nCol + 1, 0));
        // dp[i][j] represents max square ending at position (i-1, j-1)
        // dp[i][j] = min(dp[i - 1][j], dp[i][j - 1], dp[i - 1][j - 1]) + 1
        for(int i = 1; i <= nRow; ++i){
            for(int j = 1; j <= nCol; ++j){
                if(matrix[i-1][j-1] == '1'){
                    dp[i][j] = min(min(dp[i][j-1],dp[i-1][j]), dp[i-1][j-1]) + 1;
                    maxSq = max(maxSq, dp[i][j]);
                }   
            }
        }
        return maxSq * maxSq;      
    }
};
\end{lstlisting}


\section{Maximal Rectangle (H)}
Given a 2D binary matrix filled with 0's and 1's, find the largest rectangle containing all ones and return its area.\\

For example, given the following matrix:\\
1 0 1 0 0\\
1 0 1 1 1\\
1 1 1 1 1\\
1 0 0 1 0\\
Return 6.\\
 
\begin{lstlisting}
/** The DP solution proceeds row by row, starting from the first row. 
 *  Let the maximal rectangle area at row i and column j be computed by 
 *  [right(i,j) - left(i,j)] * height(i,j).
 *  
 *  left(i,j) = max(left(i-1,j), cur_left), cur_left can be determined from the current row
 *  right(i,j) = min(right(i-1,j), cur_right), cur_right can be determined from the current row
 *  height(i,j) = height(i-1,j) + 1, if matrix[i][j]=='1';
 *  height(i,j) = 0, if matrix[i][j]=='0'
 */ 
class Solution {
public:
    int maximalRectangle(vector<vector<char>>& matrix) {
        if (matrix.empty() || matrix[0].empty()) return 0;
        int res = 0, m = matrix.size(), n = matrix[0].size();
        vector<int> height(n, 0), left(n, 0), right(n, n);
        for (int i = 0; i < m; ++i) {
            int cur_left = 0, cur_right = n;
            // compute height (can do this from either side)
            for (int j = 0; j < n; ++j) { 
                if (matrix[i][j] == '1') ++height[j];
                else height[j] = 0;
            }
            // compute left (from left to right)
            for (int j = 0; j < n; ++j) {
                if (matrix[i][j] == '1') {
                    left[j] = max(left[j], cur_left);
                }
                else {
                    left[j] = 0; 
                    cur_left = j + 1;
                }
            }
            // compute right (from right to left)
            for (int j = n - 1; j >= 0; --j) {
                if (matrix[i][j] == '1'){ 
                    right[j] = min(right[j], cur_right);
                }
                else {
                    right[j] = n; 
                    cur_right = j;
                }
            }
            // compute the area of rectangle (can do this from either side)
            for (int j = 0; j < n; ++j) {
                res = max(res, (right[j] - left[j]) * height[j]);
            }
        }
        return res;
    }
};
\end{lstlisting}
