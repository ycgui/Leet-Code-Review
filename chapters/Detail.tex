\chapter{Design and Implementation}
\section{Valid Sudoku (E)}
Determine if a Sudoku is valid, according to:\\
1. Each row must have the numbers 1-9 occuring just once. \\
2. Each column must have the numbers 1-9 occuring just once. \\
3. And the numbers 1-9 must occur just once in each of the 9 sub-boxes of the grid. \\

The Sudoku board could be partially filled, where empty cells are filled with the character '.'.  A valid Sudoku board (partially filled) is not necessarily solvable. Only the filled cells need to be validated. \\

\begin{lstlisting}
class Solution {
public:
    bool isValidSudoku(vector<vector<char>>& board) {
        bool used[9]; // count if the number in the cell is appeared

        for (int i = 0; i < 9; ++i) {

            // fill: Assigns val to all the elements in the range [first,last).
            // all set to false (not appear)
            fill(used, used + 9, false);
            // check rows
            for (int j = 0; j < 9; ++j) { 
                if (!check(board[i][j], used))
                    return false;
            }

            fill(used, used + 9, false);
            // check columns
            for (int j = 0; j < 9; ++j) {               
                if (!check(board[j][i], used))
                return false;
            }
        }
        
        // check 9 blocks
        for (int m = 0; m < 3; ++m) {                   
            for (int n = 0; n < 3; ++n) {
                fill(used, used + 9, false);
                for (int i = m * 3; i < m * 3 + 3; ++i) {
                    for (int j = n * 3; j < n * 3 + 3; ++j) {
                        if (!check(board[i][j], used))
                            return false;
                    }
                }
            }
        }
        
        return true;
    }
    
    bool check(char ch, bool used[9]) {
        if (ch == '.')      return true; // continue if the current cell is .
        if (used[ch - '1']) return false; // not first time appear, return false
        return used[ch - '1'] = true; // first time appear, return true
    }
};
\end{lstlisting}


\section{Sudoku Solver (H)}
Write a program to solve a Sudoku puzzle by filling the empty cells. Empty cells are indicated by the character '.'. You may assume that there will be only one unique solution. \\

\begin{lstlisting}
class Solution {
public:
    bool solveSudoku(vector<vector<char> > &board) {
        for (int i = 0; i < 9; ++i) {
            for (int j = 0; j < 9; ++j) {
                if (board[i][j] == '.') {
                    for (int k = 0; k < 9; ++k) {
                        board[i][j] = '1' + k;
                        if (isValid(board, i, j) && solveSudoku(board))
                            return true;
                        board[i][j] = '.';
                    }
                    return false;
                }
            }
        }
        return true;
    }

    bool isValid(const vector<vector<char> > &board, int x, int y) {
        int i, j;
        for (i = 0; i < 9; i++) {  
              if (i != x && board[i][y] == board[x][y])
                  return false;
        }
        for (j = 0; j < 9; j++) {  
                if (j != y && board[x][j] == board[x][y])
                  return false;
        }
        for (i = 3 * (x / 3); i < 3 * (x / 3 + 1); i++) {
            for (j = 3 * (y / 3); j < 3 * (y / 3 + 1); j++) {
                if ((i != x || j != y) && board[i][j] == board[x][y])
                    return false;
            }
        }
        return true;
    }
};
\end{lstlisting}


\section{Design Tic-Tac-Toe (M)}
Design a Tic-tac-toe game that is played between two players on a n x n grid. \\

You may assume the following rules:\\
A move is guaranteed to be valid and is placed on an empty block.\\
Once a winning condition is reached, no more moves is allowed.\\
A player who succeeds in placing n of their marks in a horizontal, vertical, or diagonal row wins the game.\\

Example:\\
Given n = 3, assume that player 1 is "X" and player 2 is "O" in the board.\\
TicTacToe toe = new TicTacToe(3);\\

toe.move(0, 0, 1); -> Returns 0 (no one wins)\\
|X| | |\\
| | | | // Player 1 makes a move at (0, 0).\\
| | | |\\

toe.move(0, 2, 2); -> Returns 0 (no one wins)\\
|X| |O|\\
| | | | // Player 2 makes a move at (0, 2).\\
| | | |\\

toe.move(2, 2, 1); -> Returns 0 (no one wins)\\
|X| |O|\\
| | | | // Player 1 makes a move at (2, 2).\\
| | |X|\\

toe.move(1, 1, 2); -> Returns 0 (no one wins)\\
|X| |O|\\
| |O| | // Player 2 makes a move at (1, 1).\\
| | |X|\\

toe.move(2, 0, 1); -> Returns 0 (no one wins)\\
|X| |O|\\
| |O| | // Player 1 makes a move at (2, 0).\\
|X| |X|\\

toe.move(1, 0, 2); -> Returns 0 (no one wins)\\
|X| |O|\\
|O|O| | // Player 2 makes a move at (1, 0).\\
|X| |X|\\

toe.move(2, 1, 1); -> Returns 1 (player 1 wins)\\
|X| |O|\\
|O|O| | // Player 1 makes a move at (2, 1).\\
|X|X|X|\\

Follow up:\\
Could you do better than $O(n^2)$ per move() operation?\\

Hint:\\
Could you trade extra space such that move() operation can be done in O(1)? You need two arrays: int rows[n], int cols[n], plus two variables: diagonal, antidiagonal. \\

\begin{lstlisting}
class TicTacToe {
public:
    /** Initialize your data structure here. */
    TicTacToe(int n) {
        board.resize(n, vector<int>(n, 0));   
    }

    int move(int row, int col, int player) {
        board[row][col] = player;
        int i = 0, j = 0, N = board.size();
        // check row
        for (i = 0; i < N; ++i) {
            if (board[i][0] != 0) {
                for (j = 1; j < N; ++j) {
                    if (board[i][j] != board[i][j - 1]) break;
                }
                if (j == N) return board[i][0];
            }
        }
        // check column
        for (j = 0; j < N; ++j) {
            if (board[0][j] != 0) {
                for (i = 1; i < N; ++i) {
                    if (board[i][j] != board[i - 1][j]) break;
                }
                if (i == N) return board[0][j];
            }
        }
        // check diagonal
        if (board[0][0] != 0) {
            for (i = 1; i < N; ++i) {
                if (board[i][i] != board[i - 1][i - 1]) break;
            }
            if (i == N) return board[0][0];
        }
        // check antidiagonal
        if (board[N - 1][0] != 0) {
            for (i = 1; i < N; ++i) {
                if (board[N - i - 1][i] != board[N - i][i - 1]) break;
            }
            if (i == N) return board[N - 1][0];
        }
        return 0;
    }
    
private:
    vector<vector<int>> board;
};
\end{lstlisting}


\section{Boom Enemy (M)}
Given a 2D grid, each cell is either a wall 'W', an enemy 'E' or empty '0' (the number zero), return the maximum enemies you can kill using one bomb.
The bomb kills all the enemies in the same row and column from the planted point until it hits the wall since the wall is too strong to be destroyed.
Note that you can only put the bomb at an empty cell.\\

Example:\\
For the given grid\\
0 E 0 0\\
E 0 W E\\
0 E 0 0\\
return 3. (Placing a bomb at (1,1) kills 3 enemies)\\

\begin{lstlisting}
class Solution {
public:
    int maxKilledEnemies(vector<vector<char>>& grid) {
        if (grid.empty() || grid[0].empty()) return 0;
        int m = grid.size(), n = grid[0].size(), res = 0, tmp;
        vector<vector<int>> v1(m, vector<int>(n, 0)), v2 = v1, v3 = v1, v4 = v1;
        // search each row
        for (int i = 0; i < m; ++i) { 
            for (int j = 0; j < n; ++j) { // from left to right
                if (j == 0 || grid[i][j] == 'W') tmp = 0;
                else tmp = v1[i][j-1];
                if (grid[i][j] == 'E') v1[i][j] = tmp + 1;
                else v1[i][j] = tmp;
            }
            for (int j = n - 1; j >= 0; --j) { // from right to left
                if (j == n - 1 || grid[i][j] == 'W') tmp = 0;
                else tmp = v2[i][j+1];
                if (grid[i][j] == 'E') v2[i][j] = tmp + 1;
                else v2[i][j] = tmp;
            }
        }
        // search each column
        for (int j = 0; j < n; ++j) { 
            for (int i = 0; i < m; ++i) { // from up to bottom
                if (i == 0 || grid[i][j] == 'W') tmp = 0;
                else tmp = v3[i-1][j];
                if (grid[i][j] == 'E') v3[i][j] = tmp + 1;
                else v3[i][j] = tmp;
            }
            for (int i = m - 1; i >= 0; --i) { // from bottom to up
                if (i == m - 1 || grid[i][j] == 'W') tmp = 0;
                else tmp = v4[i+1][j];
                if (grid[i][j] == 'E') v4[i][j] = tmp + 1;
                else v4[i][j] = tmp;
            }
        }
        // iterate all empty cell and get the max 
        for (int i = 0; i < m; ++i) {
            for (int j = 0; j < n; ++j) {
                if (grid[i][j] == '0') {
                    res = max(res, v1[i][j] + v2[i][j] + v3[i][j] + v4[i][j]);
                }
            }
        }
        return res;
    }
};
\end{lstlisting}


\section{Android Unlock Patterns (M)}
Given an Android 3x3 key lock screen and two integers m and n, where $1 \leq m \leq n \leq 9$, count the total number of unlock patterns of the Android lock screen, which consist of minimum of m keys and maximum n keys.\\

Rules for a valid pattern:\\
    Each pattern must connect at least m keys and at most n keys.\\
    All the keys must be distinct.\\
    If the line connecting two consecutive keys in the pattern passes through any other keys, the other keys must have previously selected in the pattern. No jumps through non selected key is allowed.\\
    The order of keys used matters.\\

\begin{lstlisting}
class Solution {
public:
    int numberOfPatterns(int m, int n) {
        int res = 0;
        vector<bool> visited(10, false);
        vector<vector<int>> jumps(10, vector<int>(10, 0));
        jumps[1][3] = jumps[3][1] = 2;
        jumps[4][6] = jumps[6][4] = 5;
        jumps[7][9] = jumps[9][7] = 8;
        jumps[1][7] = jumps[7][1] = 4;
        jumps[2][8] = jumps[8][2] = 5;
        jumps[3][9] = jumps[9][3] = 6;
        jumps[1][9] = jumps[9][1] = jumps[3][7] = jumps[7][3] = 5;
        res += helper(1, 1, 0, m, n, jumps, visited) * 4;
        res += helper(2, 1, 0, m, n, jumps, visited) * 4;
        res += helper(5, 1, 0, m, n, jumps, visited);
        return res;
    }
    int helper(int num, int len, int res, int m, int n, vector<vector<int>> &jumps, vector<bool> &visited) {
        if (len >= m) ++res;
        ++len;
        if (len > n) return res;
        visited[num] = true;
        for (int next = 1; next <= 9; ++next) {
            int jump = jumps[num][next];
            if (!visited[next] && (jump == 0 || visited[jump])) {
                res = helper(next, len, res, m, n, jumps, visited);
            }
        }
        visited[num] = false;
        return res;
    }
};
\end{lstlisting}


\section{Game of Life (M)}
According to the Wikipedia's article: "The Game of Life, also known simply as Life, is a cellular automaton devised by the British mathematician John Horton Conway in 1970." Given a board with m by n cells, each cell has an initial state live (1) or dead (0). Each cell interacts with its eight neighbors (horizontal, vertical, diagonal) using the following four rules (taken from the above Wikipedia article):\\

    Any live cell with fewer than two live neighbors dies, as if caused by under-population.\\
    Any live cell with two or three live neighbors lives on to the next generation.\\
    Any live cell with more than three live neighbors dies, as if by over-population.\\
    Any dead cell with exactly three live neighbors becomes a live cell, as if by reproduction.\\

Write a function to compute the next state (after one update) of the board given its current state.\\

Follow up:\\
    Could you solve it in-place? Remember that the board needs to be updated at the same time: You cannot update some cells first and then use their updated values to update other cells.\\
    In this question, we represent the board using a 2D array. In principle, the board is infinite, which would cause problems when the active area encroaches the border of the array. How would you address these problems?\\
    
\begin{lstlisting}
/** [2nd bit, 1st bit] = [next state, current state]
 * 00  dead (next) <- dead (current)
 * 01  dead (next) <- live (current)  
 * 10  live (next) <- dead (current)  
 * 11  live (next) <- live (current) 
 * 
 * 1. In the beginning, every cell is either 00 or 01.
 * 2. Notice that 1st state is independent of 2nd state.
 * 3. Imagine all cells are instantly changing from the 1st to the 2nd state, at the same time.
 * 4. Let's count # of neighbors from 1st state and set 2nd state bit.
 * 5. Since every 2nd state is by default dead, no need to consider transition 01 -> 00.
 * 6. In the end, delete every cell's 1st state by doing >> 1.
 * 
 * For each cell's 1st bit, check the 8 pixels around itself, and set the cell's 2nd bit.
    Transition 01 -> 11: when board == 1 and (lives == 2 || lives == 3).
    Transition 00 -> 10: when board == 0 and lives == 3.
 * To get the current state, simply do board[i][j] & 1
 * To get the next state, simply do board[i][j] >> 1
 */
class Solution {
public:
    void gameOfLife(vector<vector<int>>& board) {
        if (board.empty() || board.size() == 0) return;
        int m = board.size(), n = board[0].size();
        for (int i = 0; i < m; ++i) {
            for (int j = 0; j < n; ++j) {
                int lives = getLives(board, i, j, m, n); // get current live numbers
                if (board[i][j] == 1 && (lives == 2 || lives == 3)) {
                    board[i][j] = 3;            // 01 -> 11
                }
                
                if (board[i][j] == 0 && lives == 3) {
                    board[i][j] = 2;            // 00 -> 10
                }
            }
        }
        // update board based on the 2nd bit
        for (int i = 0; i < m; ++i) {
            for (int j = 0; j < n; ++j) {
                board[i][j] >>= 1; // get the 2nd bit
            }
        }
    }
    int getLives(vector<vector<int>>& board, int i, int j, int m, int n) {
        int lives = 0;
        for (int p = max(i-1, 0); p <= min(i+1, m-1); ++p) {
            for (int q = max(j-1, 0); q <= min(j+1, n-1); ++q) {
                lives += board[p][q] & 1;       // add the 1st bit
            }
        }
        lives -= board[i][j] & 1;               // remove (i,j) itself
        return lives;
    }
};    
\end{lstlisting}


\section{Peeking Iterator (M)}
Given an Iterator class interface with methods: next() and hasNext(), design and implement a PeekingIterator that support the peek() operation -- it essentially peek() at the element that will be returned by the next call to next().\\

Here is an example. Assume that the iterator is initialized to the beginning of the list: [1, 2, 3].\\
Call next() gets you 1, the first element in the list.\\
Now you call peek() and it returns 2, the next element. Calling next() after that still return 2.\\
You call next() the final time and it returns 3, the last element. Calling hasNext() after that should return false.\\

Hint:\\
    Think of "looking ahead". You want to cache the next element.\\
    Is one variable sufficient? Why or why not?\\
    Test your design with call order of peek() before next() vs next() before peek().\\
    For a clean implementation, check out Google's guava library source code. \\

Follow up: How would you extend your design to be generic and work with all types, not just integer?\\

\begin{lstlisting}
// Below is the interface for Iterator, which is already defined for you.
// **DO NOT** modify the interface for Iterator.
class Iterator {
    struct Data;
	Data* data;
public:
	Iterator(const vector<int>& nums);
	Iterator(const Iterator& iter);
	virtual ~Iterator();
	// Returns the next element in the iteration.
	int next();
	// Returns true if the iteration has more elements.
	bool hasNext() const;
};


class PeekingIterator : public Iterator {
public:
	PeekingIterator(const vector<int>& nums) : Iterator(nums) {
	    // Initialize any member here.
	    // **DO NOT** save a copy of nums and manipulate it directly.
	    // You should only use the Iterator interface methods.
	    flag = false; // use flag to indict if next is saved in value
	}

    // Returns the next element in the iteration without advancing the iterator.
	int peek() {
        if (!flag) {
            value = Iterator::next();
            flag = true;
        }
        return value;
	}

	// hasNext() and next() should behave the same as in the Iterator interface.
	// Override them if needed.
	int next() {
	    if (!flag) {
	        return Iterator::next();
	    } else {
    	    flag = false;
	        return value;
	    }
	}

	bool hasNext() const {
	    if (flag || Iterator::hasNext()) return true;
	    else return false;
	}
	
private:
    bool flag;
    int value;
};
\end{lstlisting}


\section{Design Phone Directory (M)}
Design a Phone Directory which supports the following operations:\\
    get: Provide a number which is not assigned to anyone.\\
    check: Check if a number is available or not.\\
    release: Recycle or release a number.\\

Example:\\
// Init a phone directory containing a total of 3 numbers: 0, 1, and 2.\\
PhoneDirectory directory = new PhoneDirectory(3);\\

// It can return any available phone number. Here we assume it returns 0.\\
directory.get();\\

// Assume it returns 1.\\
directory.get();\\

// The number 2 is available, so return true.\\
directory.check(2);\\

// It returns 2, the only number that is left.\\
directory.get();\\

// The number 2 is no longer available, so return false.\\
directory.check(2);\\

// Release number 2 back to the pool.\\
directory.release(2);\\

// Number 2 is available again, return true.\\
directory.check(2);\\

\begin{lstlisting}
class PhoneDirectory {
public:
    /** Initialize your data structure here
        @param maxNumbers - The maximum numbers that can be stored in the phone directory. */
    PhoneDirectory(int maxNumbers) {
        max_num = maxNumbers;
        next = idx = 0;
        recycle.resize(max_num);
        flag.resize(max_num, 1);
    }
    
    /** Provide a number which is not assigned to anyone.
        @return - Return an available number. Return -1 if none is available. */
    int get() {
        if (next == max_num && idx <= 0) return -1;
        if (idx > 0) {
            int t = recycle[--idx];
            flag[t] = 0;
            return t;
        }
        flag[next] = false;
        return next++;
    }
    
    /** Check if a number is available or not. */
    bool check(int number) {
        return number >= 0 && number < max_num && flag[number];
    }
    
    /** Recycle or release a number. */
    void release(int number) {
        if (number >= 0 && number < max_num && !flag[number]) {
            recycle[idx++] = number;
            flag[number] = 1;
        }
    }
private:
    int max_num, next, idx;
    vector<int> recycle, flag;
};
\end{lstlisting}


\section{Insert Delete GetRandom O(1) (H)}
Design a data structure that supports all following operations in average O(1) time.\\
    insert(val): Inserts an item val to the set if not already present.\\
    remove(val): Removes an item val from the set if present.\\
    getRandom: Returns a random element from current set of elements. Each element must have the same probability of being returned.\\

Example:\\
// Init an empty set.\\
RandomizedSet randomSet = new RandomizedSet();\\

// Inserts 1 to the set. Returns true as 1 was inserted successfully.\\
randomSet.insert(1);\\

// Returns false as 2 does not exist in the set.\\
randomSet.remove(2);\\

// Inserts 2 to the set, returns true. Set now contains [1,2].\\
randomSet.insert(2);\\

// getRandom should return either 1 or 2 randomly.\\
randomSet.getRandom();\\

// Removes 1 from the set, returns true. Set now contains [2].\\
randomSet.remove(1);\\

// 2 was already in the set, so return false.\\
randomSet.insert(2);\\

// Since 1 is the only number in the set, getRandom always return 1.\\
randomSet.getRandom();\\

\begin{lstlisting}
class RandomizedSet {
public:
    /** Initialize your data structure here. */
    RandomizedSet() {}
    
    /** Inserts a value to the set. Returns true if the set did not already contain the specified element. */
    bool insert(int val) {
        if (m.count(val)) return false;
        nums.push_back(val);
        m[val] = nums.size()-1; // index is the last index of nums
        return true;
    }
    
    /** Removes a value from the set. Returns true if the set contained the specified element. */
    // switch val and last in nums, then remove the end of nums
    bool remove(int val) {
        if (!m.count(val)) return false;
        int last = nums.back(); // save the last element of nums
        m[last] = m[val]; // update the index of last to the index of val
        nums[m[val]] = last; // overwrite val in nums by last
        nums.pop_back(); // remove the end in nums
        m.erase(val); // remove val in m
        return true;
    }
    
    /** Get a random element from the set. */
    int getRandom() {
        return nums[rand() % nums.size()];
    }
private:
    vector<int> nums;
    unordered_map<int, int> m;
};

/**
 * Your RandomizedSet object will be instantiated and called as such:
 * RandomizedSet obj = new RandomizedSet();
 * bool param_1 = obj.insert(val);
 * bool param_2 = obj.remove(val);
 * int param_3 = obj.getRandom();
 */
\end{lstlisting}


\section{Insert Delete GetRandom O(1) - Duplicates allowed ? (H)}
Design a data structure that supports all following operations in average O(1) time.\\
Note: Duplicate elements are allowed.\\
    insert(val): Inserts an item val to the collection.\\
    remove(val): Removes an item val from the collection if present.\\
    getRandom: Returns a random element from current collection of elements. The probability of each element being returned is linearly related to the number of same value the collection contains.\\

Example:\\
// Init an empty collection.\\
RandomizedCollection collection = new RandomizedCollection();\\

// Inserts 1 to the collection. Returns true as the collection did not contain 1.\\
collection.insert(1);\\

// Inserts another 1 to the collection. Returns false as the collection contained 1. Collection now contains [1,1].\\
collection.insert(1);\\

// Inserts 2 to the collection, returns true. Collection now contains [1,1,2].\\
collection.insert(2);\\

// getRandom should return 1 with the probability 2/3, and returns 2 with the probability 1/3.\\
collection.getRandom();\\

// Removes 1 from the collection, returns true. Collection now contains [1,2].\\
collection.remove(1);\\

// getRandom should return 1 and 2 both equally likely.\\
collection.getRandom();\\

\begin{lstlisting}
class RandomizedCollection {
public:
    /** Initialize your data structure here. */
    RandomizedCollection() {}
    
    /** Inserts a value to the collection. Returns true if the collection did not already contain the specified element. */
    bool insert(int val) {
        m[val].push_back(nums.size());
        nums.push_back(val);
        return m[val].size() == 1;
    }
    
    /** Removes a value from the collection. Returns true if the collection contained the specified element. */
    bool remove(int val) {
        if (!m.count(val)) return false;
        int last = nums.back(), pos = m[val].back(); 
        m[last].back() = m[val].back();
        if (m[val].size() > 1) m[val].pop_back();
        else m.erase(val);
        nums.pop_back();
        nums[pos] = last;
        return true;
    }
    
    /** Get a random element from the collection. */
    int getRandom() {
        return nums[rand() % nums.size()];
    }
    
private:
    vector<int> nums;
    unordered_map<int, vector<int>> m;
};

/**
 * Your RandomizedCollection object will be instantiated and called as such:
 * RandomizedCollection obj = new RandomizedCollection();
 * bool param_1 = obj.insert(val);
 * bool param_2 = obj.remove(val);
 * int param_3 = obj.getRandom();
 */
\end{lstlisting}


\section{LRU Cache (H)}
Design and implement a data structure for Least Recently Used (LRU) cache. It should support the following operations: get and set.\\

get(key) - Get the value (will always be positive) of the key if the key exists in the cache, otherwise return -1.\\
set(key, value) - Set or insert the value if the key is not already present. When the cache reached its capacity, it should invalidate the least recently used item before inserting a new item. \\

\begin{lstlisting}
class LRUCache{
public:
    LRUCache(int capacity) {
        cap = capacity;
    }
    
    int get(int key) {
        auto it = m.find(key);
        if (it == m.end()) return -1;
        // void splice (iterator position, list& x, iterator i)
        // transfers only the element pointed by i from x into the container, inserting it at position.
        l.splice(l.begin(), l, it->second);
        return it->second->second;
    }
    
    void set(int key, int value) {
        auto it = m.find(key);
        // if the current key and value exist in list, remove it
        if (it != m.end()) l.erase(it->second);
        // insert the new pair to the top of list
        l.push_front(make_pair(key, value)); 
        m[key] = l.begin();
        // if capacity is exceeded, remove the bottom one
        if (m.size() > cap) {
            int k = l.rbegin()->first;
            l.pop_back();
            m.erase(k);
        }
    }
    
private:
    int cap;
    list<pair<int, int> > l;
    unordered_map<int, list<pair<int, int> >::iterator> m;
};
\end{lstlisting}


\section{Basic Calculator (H)}
Implement a basic calculator to evaluate a simple expression string. The expression string may contain open ( and closing parentheses ), the plus + or minus sign -, non-negative integers and empty spaces . You may assume that the given expression is always valid.\\

Some examples:\\
$"1 + 1" = 2$\\
$" 2-1 + 2 " = 3$\\
$"(1+(4+5+2)-3)+(6+8)" = 23$\\

Note: Do not use the eval built-in library function. \\

\begin{lstlisting}
class Solution {
public:
    int calculate(string s) {
        int res = 0;
        vector<int> sign(2, 1); // use sign to decide add or minus
        for (int i = 0; i < s.size(); ++i) {
            char c = s[i];
            if (c >= '0') {
                int num = 0;
                while (i < s.size() && s[i] >= '0') {
                    num = 10 * num + s[i++] - '0'; // get the value
                }
                res += sign.back() * num;
                sign.pop_back();
                --i;
            }
            else if (c == ')') sign.pop_back();
            else if (c != ' ') sign.push_back(sign.back() * (c == '-' ? -1 : 1));
        }
        return res;
    }
};
\end{lstlisting}


\section{Basic Calculator II (M)}
Implement a basic calculator to evaluate a simple expression string. The expression string contains only non-negative integers, +, -, *, / operators and empty spaces . The integer division should truncate toward zero. You may assume that the given expression is always valid.

Some examples:\\
$"3+2*2" = 7$\\
$" 3/2 " = 1$\\
$" 3+5 / 2 " = 5$\\

Note: Do not use the eval built-in library function. \\

\begin{lstlisting}
class Solution {
public:
    int calculate(string s) {
        int res = 0, d = 0;
        char sign = '+';
        stack<int> nums;
        for (int i = 0; i < s.size(); ++i) {
            if (s[i] >= '0') {
                d = d * 10 + s[i] - '0';
            }
            if ((s[i] < '0' && s[i] != ' ') || i == s.size() - 1) {
                if (sign == '+') nums.push(d);
                if (sign == '-') nums.push(-d);
                if (sign == '*' || sign == '/') {
                    int tmp = sign == '*' ? nums.top() * d : nums.top() / d;
                    nums.pop();
                    nums.push(tmp);
                }
                sign = s[i];
                d = 0;
            } 
        }
        while (!nums.empty()) {
            res += nums.top();
            nums.pop();
        }
        return res;
    }
};
\end{lstlisting}

