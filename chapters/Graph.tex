\chapter{Graph}
\section{Number of Connected Components in an Undirected Graph (M)}
Given n nodes labeled from 0 to n - 1 and a list of undirected edges (each edge is a pair of nodes), write a function to find the number of connected components in an undirected graph.\\

Example 1: 
Given n = 5 and edges = [[0, 1], [1, 2], [3, 4]], return 2.\\

Example 2:
Given n = 5 and edges = [[0, 1], [1, 2], [2, 3], [3, 4]], return 1.\\

Note:
You can assume that no duplicate edges will appear in edges. Since all edges are undirected, [0, 1] is the same as [1, 0] and thus will not appear together in edges.\\

\begin{lstlisting}
class Solution {
public:
    int countComponents(int n, vector<pair<int, int> >& edges) {
        vector<vector<int>> graph(n);   // adjacency list
        vector<bool> visit(n, false);   // visit state
        int res = 0;
        // create the adjacency list
        for (auto edge : edges) {
            graph[edge.first].push_back(edge.second);
            graph[edge.second].push_back(edge.first);
        }
        for (int i = 0; i < n; ++i) {
            if (!visit[i]) {	// if i is not visited
                ++res;	// increase the number of connected components
                dfs(graph, visit, i);   // use dfs to visit all nodes in the current conneted component
            }
        }
        return res;
    }
    
    void dfs(vector<vector<int>> &graph, vector<bool> &visit, int i) {
        if (visit[i])   return;
        visit[i] = true;
        for (int j = 0; j < graph[i].size(); ++j) {
            dfs(graph, visit, graph[i][j]);     // traverse all nodes connected to i
        }
    }
};
\end{lstlisting}


\section{Graph Valid Tree (M)}
Given n nodes labeled from 0 to n - 1 and a list of undirected edges (each edge is a pair of nodes), write a function to check whether these edges make up a valid tree.\\

For example:\\
Given n = 5 and edges = [[0, 1], [0, 2], [0, 3], [1, 4]], return true.\\
Given n = 5 and edges = [[0, 1], [1, 2], [2, 3], [1, 3], [1, 4]], return false.\\

Hint:\\
    Given n = 5 and edges = [[0, 1], [1, 2], [3, 4]], what should your return? Is this case a valid tree?\\
    According to the definition of tree on Wikipedia: ?a tree is an undirected graph in which any two vertices are connected by exactly one path. In other words, any connected graph without simple cycles is a tree.?\\

Note: you can assume that no duplicate edges will appear in edges. Since all edges are undirected, [0, 1] is the same as [1, 0] and thus will not appear together in edges.\\

\begin{lstlisting}
// DFS
class Solution {
public:
    bool validTree(int n, vector<pair<int, int>>& edges) {
        vector<vector<int>> graph(n, vector<int>());
        vector<bool> visit(n, false);
        for (auto edge : edges) {
            g[edge.first].push_back(edge.second);
            g[edge.second].push_back(edge.first);
        }
        if (!dfs(graph, visit, 0, -1)) return false;
        for (auto v : visit) {
            if (!v) return false;
        }
        return true;
    }
    bool dfs(vector<vector<int>> &graph, vector<bool> &visit, int cur, int pre) {
        if (visit[cur]) return false;
        visit[cur] = true;
        for (auto g : graph[cur]) {
            if (g != pre) {
                if (!dfs(graph, visit, g, cur)) return false;
            }
        }
        return true;
    }
};

// BFS
class Solution {
public:
    bool validTree(int n, vector<pair<int, int>>& edges) {
        vector<unordered_set<int>> graph(n, unordered_set<int>());
        unordered_set<int> visit;
        queue<int> q;
        q.push(0);
        visit.insert(0);
        for (auto edge : edges) {
            graph[edge.first].insert(edge.second);
            graph[edge.second].insert(edge.first);
        }
        while (!q.empty()) {
            int t = q.front(); q.pop();
            for (auto g : graph[t]) {
                if (visit.find(g) != visit.end()) return false;
                visit.insert(g);
                q.push(g);
                graph[g].erase(t);
            }
        }
        return visit.size() == n;
    }
};
\end{lstlisting}


\section{Clone Graph (M)}
Clone an undirected graph. Each node in the graph contains a label and a list of its neighbors. \\

\begin{lstlisting}
/**
 * Definition for undirected graph.
 * struct UndirectedGraphNode {
 *     int label;
 *     vector<UndirectedGraphNode *> neighbors;
 *     UndirectedGraphNode(int x) : label(x) {};
 * };
 */
class Solution {
public:
    UndirectedGraphNode *cloneGraph(UndirectedGraphNode *node) {
        // use hashtable to map label and neighbors
        unordered_map<int, UndirectedGraphNode*> umap;
        return clone(node, umap);
    }
    UndirectedGraphNode *clone(UndirectedGraphNode *node, unordered_map<int, UndirectedGraphNode*> &umap) {
        if (!node) return node;
        // if label is in the hashtable, return the list of neighbors
        if (umap.count(node->label)) return umap[node->label];
        // otherwise, define a new node that has label
        UndirectedGraphNode *newNode = new UndirectedGraphNode(node->label);
        // update hashtable
        umap[node->label] = newNode;
        // perform DFS
        // create a list of neighbors for the new node
        for (int i = 0; i < node->neighbors.size(); ++i) {
            (newNode->neighbors).push_back(clone(node->neighbors[i], umap));
        }
        return newNode;
    } 
};
\end{lstlisting}


\section{Number of Islands (M)}
Given a 2d grid map of '1's (land) and '0's (water), count the number of islands. An island is surrounded by water and is formed by connecting adjacent lands horizontally or vertically. You may assume all four edges of the grid are all surrounded by water.\\

Example 1:\\
11110\\
11010\\
11000\\
00000\\
Answer: 1\\

Example 2:\\
11000\\
11000\\
00100\\
00011\\
Answer: 3\\

\begin{lstlisting}
class Solution {
public:
    int numIslands(vector<vector<char> > &grid) {
        if (grid.empty() || grid[0].empty()) return 0;
        int m = grid.size(), n = grid[0].size(), res = 0;
        vector<vector<bool> > visited(m, vector<bool>(n, false));
        for (int i = 0; i < m; ++i) {
            for (int j = 0; j < n; ++j) {
                if (grid[i][j] == '1' && !visited[i][j]) {
                    numIslandsDFS(grid, visited, i, j);
                    ++res;
                }
            }
        }
        return res;
    }
    void numIslandsDFS(vector<vector<char> > &grid, vector<vector<bool> > &visited, int x, int y) {
        if (x < 0 || x >= grid.size()) return;
        if (y < 0 || y >= grid[0].size()) return;
        if (grid[x][y] != '1' || visited[x][y]) return;
        visited[x][y] = true; // grid[x][y] is visited now
        // search other 4 direction of grid[x][y]
        numIslandsDFS(grid, visited, x - 1, y);
        numIslandsDFS(grid, visited, x + 1, y);
        numIslandsDFS(grid, visited, x, y - 1);
        numIslandsDFS(grid, visited, x, y + 1);
    }
};
\end{lstlisting}


\section{Number of Islands II (H)}
A 2d grid map of m rows and n columns is initially filled with water. We may perform an addLand operation which turns the water at position (row, col) into a land. Given a list of positions to operate, count the number of islands after each addLand operation. An island is surrounded by water and is formed by connecting adjacent lands horizontally or vertically. You may assume all four edges of the grid are all surrounded by water. \\

Example:\\
Given m = 3, n = 3, positions = [[0,0], [0,1], [1,2], [2,1]].\\

Initially, the 2d grid grid is filled with water. (Assume 0 represents water and 1 represents land).\\
0 0 0\\
0 0 0\\
0 0 0\\

Operation \#1: addLand(0, 0) turns the water at grid[0][0] into a land.\\
1 0 0\\
0 0 0   Number of islands = 1\\
0 0 0\\

Operation \#2: addLand(0, 1) turns the water at grid[0][1] into a land.\\
1 1 0\\
0 0 0   Number of islands = 1\\
0 0 0\\

Operation \#3: addLand(1, 2) turns the water at grid[1][2] into a land.\\
1 1 0\\
0 0 1   Number of islands = 2\\
0 0 0\\

Operation \#4: addLand(2, 1) turns the water at grid[2][1] into a land.\\
1 1 0\\
0 0 1   Number of islands = 3\\
0 1 0\\

We return the result as an array: [1, 1, 2, 3]\\

Challenge: Can you do it in time complexity O(k log mn), where k is the length of the positions?\\

\begin{lstlisting}
class Solution {
public:
    vector<int> numIslands2(int m, int n, vector<pair<int, int>>& positions) {
        vector<int> res;
        if (m <= 0 || n <= 0) return res;
        vector<int> roots(m * n, -1);
        int cnt = 0;
        vector<vector<int> > dirs{{0, -1}, {-1, 0}, {0, 1}, {1, 0}};
        for (auto a : positions) {
            int id = n * a.first + a.second;
            roots[id] = id;
            ++cnt;
            for (auto d : dirs) {
                int x = a.first + d[0], y = a.second + d[1];
                int cur_id = n * x + y;
                if (x < 0 || x >= m || y < 0 || y >= n || roots[cur_id] == -1) continue;
                int new_id = findRoots(roots, cur_id);
                if (id != new_id) {
                    roots[id] = new_id;
                    id = new_id;
                    --cnt;
                }
            }
            res.push_back(cnt);
        }
        return res;
    }
    int findRoots(vector<int> &roots, int id) {
        while (id != roots[id]) {
            roots[id] = roots[roots[id]];
            id = roots[id];
        }
        return id;
    }
};
\end{lstlisting}
