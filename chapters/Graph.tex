\chapter{Graph}
\section{Number of Connected Components in an Undirected Graph (M)}
Given n nodes labeled from 0 to n - 1 and a list of undirected edges (each edge is a pair of nodes), write a function to find the number of connected components in an undirected graph.\\

Example 1: 
Given n = 5 and edges = [[0, 1], [1, 2], [3, 4]], return 2.\\

Example 2:
Given n = 5 and edges = [[0, 1], [1, 2], [2, 3], [3, 4]], return 1.\\

Note:
You can assume that no duplicate edges will appear in edges. Since all edges are undirected, [0, 1] is the same as [1, 0] and thus will not appear together in edges.\\

\begin{lstlisting}
class Solution {
public:
    int countComponents(int n, vector<pair<int, int> >& edges) {
        vector<vector<int>> graph(n);   // adjacency list
        vector<bool> visit(n, false);   // visit state
        int res = 0;
        // create the adjacency list
        for (auto edge : edges) {
            graph[edge.first].push_back(edge.second);
            graph[edge.second].push_back(edge.first);
        }
        for (int i = 0; i < n; ++i) {
            if (!visit[i]) {	// if i is not visited
                ++res;	// increase the number of connected components
                dfs(graph, visit, i);   // use dfs to visit all nodes in the current conneted component
            }
        }
        return res;
    }
    
    void dfs(vector<vector<int>> &graph, vector<bool> &visit, int i) {
        if (visit[i])   return;
        visit[i] = true;
        for (int j = 0; j < graph[i].size(); ++j) {
            dfs(graph, visit, graph[i][j]);     // traverse all nodes connected to i
        }
    }
};
\end{lstlisting}


\section{Graph Valid Tree (M)}
Given n nodes labeled from 0 to n - 1 and a list of undirected edges (each edge is a pair of nodes), write a function to check whether these edges make up a valid tree.\\

For example:\\
Given n = 5 and edges = [[0, 1], [0, 2], [0, 3], [1, 4]], return true.\\
Given n = 5 and edges = [[0, 1], [1, 2], [2, 3], [1, 3], [1, 4]], return false.\\

Hint:\\
    Given n = 5 and edges = [[0, 1], [1, 2], [3, 4]], what should your return? Is this case a valid tree?\\
    According to the definition of tree on Wikipedia: ?a tree is an undirected graph in which any two vertices are connected by exactly one path. In other words, any connected graph without simple cycles is a tree.?\\

Note: you can assume that no duplicate edges will appear in edges. Since all edges are undirected, [0, 1] is the same as [1, 0] and thus will not appear together in edges.\\

\begin{lstlisting}
// DFS
class Solution {
public:
    bool validTree(int n, vector<pair<int, int>>& edges) {
        vector<vector<int>> graph(n, vector<int>());
        vector<bool> visit(n, false);
        for (auto edge : edges) {
            g[edge.first].push_back(edge.second);
            g[edge.second].push_back(edge.first);
        }
        if (!dfs(graph, visit, 0, -1)) return false;
        for (auto v : visit) {
            if (!v) return false;
        }
        return true;
    }
    bool dfs(vector<vector<int>> &graph, vector<bool> &visit, int cur, int pre) {
        if (visit[cur]) return false;
        visit[cur] = true;
        for (auto g : graph[cur]) {
            if (g != pre) {
                if (!dfs(graph, visit, g, cur)) return false;
            }
        }
        return true;
    }
};

// BFS
class Solution {
public:
    bool validTree(int n, vector<pair<int, int>>& edges) {
        vector<unordered_set<int>> graph(n, unordered_set<int>());
        unordered_set<int> visit;
        queue<int> q;
        q.push(0);
        visit.insert(0);
        for (auto edge : edges) {
            graph[edge.first].insert(edge.second);
            graph[edge.second].insert(edge.first);
        }
        while (!q.empty()) {
            int t = q.front(); q.pop();
            for (auto g : graph[t]) {
                if (visit.find(g) != visit.end()) return false;
                visit.insert(g);
                q.push(g);
                graph[g].erase(t);
            }
        }
        return visit.size() == n;
    }
};
\end{lstlisting}


\section{Copy List with Random Pointer (M)}
A linked list is given such that each node contains an additional random pointer which could point to any node in the list or null.\\

Return a deep copy of the list.\\

\begin{lstlisting}
"""
# Definition for a Node.
class Node(object):
    def __init__(self, val, next, random):
        self.val = val
        self.next = next
        self.random = random
"""
class Solution(object):
    def copyRandomList(self, head):
        """
        :type head: Node
        :rtype: Node
        """
        list_dict = {}
        return self.helper(head, list_dict)
    def helper(self, node, list_dict):
        if not node: return None
        # if the random node points to a node that is already in the dict,
        # then just return the node as the random node
        if node in list_dict: 
            return list_dict[node]
        new_node = Node(node.val, None, None)
        list_dict[node] = new_node
        new_node.next = self.helper(node.next, list_dict)
        new_node.random = self.helper(node.random, list_dict)
        return new_node
\end{lstlisting}


\section{Clone Graph (M)}
Clone an undirected graph. Each node in the graph contains a label and a list of its neighbors. \\

\begin{lstlisting}
"""
# Definition for a Node.
class Node(object):
    def __init__(self, val, neighbors):
        self.val = val
        self.neighbors = neighbors
"""
class Solution(object):
    def cloneGraph(self, node):
        """
        :type node: Node
        :rtype: Node
        """
        node_dict = {}
        return self.helper(node, node_dict)
    def helper(self, node, node_dict):
        if not node: return None
        if node in node_dict:
            return node_dict[node]
        new_node = Node(node.val, [])
        node_dict[node] = new_node
        for n in node.neighbors:
            new_node.neighbors.append(self.helper(n, node_dict))
        return new_node
\end{lstlisting}

\begin{lstlisting}
/**
 * Definition for undirected graph.
 * struct UndirectedGraphNode {
 *     int label;
 *     vector<UndirectedGraphNode *> neighbors;
 *     UndirectedGraphNode(int x) : label(x) {};
 * };
 */
class Solution {
public:
    UndirectedGraphNode *cloneGraph(UndirectedGraphNode *node) {
        // use hashtable to map label and neighbors
        unordered_map<int, UndirectedGraphNode*> umap;
        return clone(node, umap);
    }
    UndirectedGraphNode *clone(UndirectedGraphNode *node, unordered_map<int, UndirectedGraphNode*> &umap) {
        if (!node) return node;
        // if label is in the hashtable, return the list of neighbors
        if (umap.count(node->label)) return umap[node->label];
        // otherwise, define a new node that has label
        UndirectedGraphNode *newNode = new UndirectedGraphNode(node->label);
        // update hashtable
        umap[node->label] = newNode;
        // perform DFS
        // create a list of neighbors for the new node
        for (int i = 0; i < node->neighbors.size(); ++i) {
            (newNode->neighbors).push_back(clone(node->neighbors[i], umap));
        }
        return newNode;
    } 
};
\end{lstlisting}


\section{Course Schedule (M)}
There are a total of n courses you have to take, labeled from 0 to n - 1. Some courses may have prerequisites, for example to take course 0 you have to first take course 1, which is expressed as a pair: [0,1]. Given the total number of courses and a list of prerequisite pairs, is it possible for you to finish all courses? \\

For example:\\
2, [[1,0]]: There are a total of 2 courses to take. To take course 1 you should have finished course 0. So it is possible.\\
2, [[1,0],[0,1]]: There are a total of 2 courses to take. To take course 1 you should have finished course 0, and to take course 0 you should also have finished course 1. So it is impossible.\\

Note:
The input prerequisites is a graph represented by a list of edges, not adjacency matrices. Read more about how a graph is represented.\\

Hints:\\
    This problem is equivalent to finding if a cycle exists in a directed graph. If a cycle exists, no topological ordering exists and therefore it will be impossible to take all courses.\\
    Topological Sort via DFS - A great video tutorial (21 minutes) on Coursera explaining the basic concepts of Topological Sort.\\
    Topological sort could also be done via BFS.\\

\begin{lstlisting}
// DFS solution
class Solution {
public:
    bool canFinish(int numCourses, vector<pair<int, int>>& prerequisites) {
        vector<vector<int> > graph(numCourses, vector<int>(0));
        vector<int> visit(numCourses, 0);
        for (auto a : prerequisites) {
            graph[a.second].push_back(a.first);
        }
        for (int i = 0; i < numCourses; ++i) {
            if (!canFinishDFS(graph, visit, i)) return false;
        }
        return true;
    }
    bool canFinishDFS(vector<vector<int> > &graph, vector<int> &visit, int i) {
        if (visit[i] == -1) return false;
        if (visit[i] == 1) return true;
        visit[i] = -1;
        for (auto a : graph[i]) {
            if (!canFinishDFS(graph, visit, a)) return false;
        }
        visit[i] = 1;
        return true;
    }
};

// BFS solution
class Solution {
public:
    bool canFinish(int numCourses, vector<pair<int, int>>& prerequisites) {
        vector<vector<int> > graph(numCourses, vector<int>(0));
        vector<int> in(numCourses, 0);
        for (auto a : prerequisites) {
            graph[a.second].push_back(a.first);
            ++in[a.first];
        }
        queue<int> q;
        for (int i = 0; i < numCourses; ++i) {
            if (in[i] == 0) q.push(i);
        }
        while (!q.empty()) {
            int t = q.front();
            q.pop();
            for (auto a : graph[t]) {
                --in[a];
                if (in[a] == 0) q.push(a);
            }
        }
        for (int i = 0; i < numCourses; ++i) {
            if (in[i] != 0) return false;
        }
        return true;
    }
};
\end{lstlisting}

\section{Course Schedule II (M)}
There are a total of n courses you have to take, labeled from 0 to n - 1. Some courses may have prerequisites, for example to take course 0 you have to first take course 1, which is expressed as a pair: [0,1]. Given the total number of courses and a list of prerequisite pairs, return the ordering of courses you should take to finish all courses. There may be multiple correct orders, you just need to return one of them. If it is impossible to finish all courses, return an empty array.\\

For example:\\
2, [[1,0]]: There are a total of 2 courses to take. To take course 1 you should have finished course 0. So the correct course order is [0,1]\\
4, [[1,0],[2,0],[3,1],[3,2]]: There are a total of 4 courses to take. To take course 3 you should have finished both courses 1 and 2. Both courses 1 and 2 should be taken after you finished course 0. So one correct course order is [0,1,2,3]. Another correct ordering is[0,2,1,3].\\

Note:
The input prerequisites is a graph represented by a list of edges, not adjacency matrices. Read more about how a graph is represented.\\

\begin{lstlisting}
class Solution {
public:
    vector<int> findOrder(int numCourses, vector<pair<int, int>>& prerequisites) {
        vector<int> res;
        vector<vector<int> > graph(numCourses, vector<int>(0));
        vector<int> in(numCourses, 0);
        for (auto &a : prerequisites) {
            graph[a.second].push_back(a.first);
            ++in[a.first];
        }
        queue<int> q;
        for (int i = 0; i < numCourses; ++i) {
            if (in[i] == 0) q.push(i);
        }
        while (!q.empty()) {
            int t = q.front();
            res.push_back(t);
            q.pop();
            for (auto &a : graph[t]) {
                --in[a];
                if (in[a] == 0) q.push(a);
            }
        }
        // clear the result if a cycle exists
        if (res.size() != numCourses) res.clear();
        return res;
    }
};
\end{lstlisting}


\section{Minimum Height Trees (M)}
For a undirected graph with tree characteristics, we can choose any node as the root. The result graph is then a rooted tree. Among all possible rooted trees, those with minimum height are called minimum height trees (MHTs). Given such a graph, write a function to find all the MHTs and return a list of their root labels.\\

Format:\\
The graph contains n nodes which are labeled from 0 to n - 1. You will be given the number n and a list of undirected edges (each edge is a pair of labels). You can assume that no duplicate edges will appear in edges. Since all edges are undirected, [0, 1] is the same as [1, 0] and thus will not appear together in edges.\\

Example 1:
Given n = 4, edges = [[1, 0], [1, 2], [1, 3]], return [1]\\
Example 2:
Given n = 6, edges = [[0, 3], [1, 3], [2, 3], [4, 3], [5, 4]], return [3, 4]\\

Hint:
    How many MHTs can a graph have at most?\\
    
Note:\\
(1) According to the definition of tree on Wikipedia: ?a tree is an undirected graph in which any two vertices are connected by exactly one path. In other words, any connected graph without simple cycles is a tree.?\\
(2) The height of a rooted tree is the number of edges on the longest downward path between the root and a leaf. \\

\begin{lstlisting}
// Non-recursion
class Solution {
public:
    vector<int> findMinHeightTrees(int n, vector<pair<int, int> >& edges) {
        if (n == 1) return {0};
        vector<int> res, d(n, 0);
        vector<vector<int> > g(n, vector<int>());
        queue<int> q;
        for (auto a : edges) {
            g[a.first].push_back(a.second);
            ++d[a.first];
            g[a.second].push_back(a.first);
            ++d[a.second];
        }
        for (int i = 0; i < n; ++i) {
            if (d[i] == 1) q.push(i);
        }
        while (n > 2) {
            int sz = q.size();
            for (int i = 0; i < sz; ++i) {
                int t = q.front(); q.pop();
                --n;
                for (int i : g[t]) {
                    --d[i];
                    if (d[i] == 1) q.push(i);
                }
            }
        }
        while (!q.empty()) {
            res.push_back(q.front()); q.pop();
        }
        return res;
    }
};
\end{lstlisting}


\section{Reconstruct Itinerary (M)}
Given a list of airline tickets represented by pairs of departure and arrival airports [from, to], reconstruct the itinerary in order. All of the tickets belong to a man who departs from JFK. Thus, the itinerary must begin with JFK.\\

Note:\\
    If there are multiple valid itineraries, you should return the itinerary that has the smallest lexical order when read as a single string. For example, the itinerary ["JFK", "LGA"] has a smaller lexical order than ["JFK", "LGB"].\\
    All airports are represented by three capital letters (IATA code).\\
    You may assume all tickets form at least one valid itinerary.\\

Example 1:\\
tickets = [["MUC", "LHR"], ["JFK", "MUC"], ["SFO", "SJC"], ["LHR", "SFO"]]\\
Return ["JFK", "MUC", "LHR", "SFO", "SJC"].\\

Example 2:\\
tickets = [["JFK","SFO"],["JFK","ATL"],["SFO","ATL"],["ATL","JFK"],["ATL","SFO"]]\\
Return ["JFK","ATL","JFK","SFO","ATL","SFO"].\\
Another possible reconstruction is ["JFK","SFO","ATL","JFK","ATL","SFO"]. But it is larger in lexical order. \\

\begin{lstlisting}
class Solution {
public:
    vector<string> findItinerary(vector<pair<string, string> > tickets) {
        vector<string> res;
        unordered_map<string, multiset<string> > m;
        for (auto a : tickets) {
            m[a.first].insert(a.second);
        }
        dfs(m, "JFK", res);
        return vector<string> (res.rbegin(), res.rend());
    }
    void dfs(unordered_map<string, multiset<string> > &m, string s, vector<string> &res) {
        while (m[s].size()) {
            string t = *m[s].begin();
            m[s].erase(m[s].begin());
            dfs(m, t, res);
        }
        res.push_back(s);
    }
};
\end{lstlisting}

