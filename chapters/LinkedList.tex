\chapter{Linked List}
\section{Intersection of Two Linked Lists (E)}
Write a program to find the node at which the intersection of two singly linked lists begins.\\

Notes:\\
    If the two linked lists have no intersection at all, return null.\\
    The linked lists must retain their original structure after the function returns.\\
    You may assume there are no cycles anywhere in the entire linked structure.\\
    Your code should preferably run in O(n) time and use only O(1) memory.\\
 
 \begin{lstlisting}
# Definition for singly-linked list.
# class ListNode(object):
#     def __init__(self, x):
#         self.val = x
#         self.next = None

class Solution(object):
    def getIntersectionNode(self, headA, headB):
        """
        :type head1, head1: ListNode
        :rtype: ListNode
        """
        if not headA or not headB: return None
        p1, p2 = headA, headB
        while p1 != p2:
            p1 = p1.next
            p2 = p2.next
            if p1 == p2:
                return p1
            if not p1:
                p1 = headB
            if not p2:
                p2 = headA
        return p1
\end{lstlisting}
 
\begin{lstlisting}
/**
 * Definition for singly-linked list.
 * struct ListNode {
 *     int val;
 *     ListNode *next;
 *     ListNode(int x) : val(x), next(NULL) {}
 * };
 */
 
class Solution {
public:
    ListNode *getIntersectionNode(ListNode *headA, ListNode *headB) {
        if (!headA || !headB)   return NULL;
        
        ListNode *p1 = headA;
        ListNode *p2 = headB;
        
        while (p1 && p2 && p1 != p2) {
            p1 = p1->next;
            p2 = p2->next;
            
            if (p1 == p2)   return p1;
            
            if (!p1)    p1 = headB;
            if (!p2)    p2 = headA;
        }
        
        return p1;
    }
};
\end{lstlisting}


\section{Add Two Numbers (M)}
You are given two linked lists representing two non-negative numbers. The digits are stored in reverse order and each of their nodes contain a single digit. Add the two numbers and return it as a linked list.\\

Input: $(2 -> 4 -> 3) + (5 -> 6 -> 4)$ \\
Output: $7 -> 0 -> 8$\\

Python:
\lstset{language=python}
\begin{lstlisting}
# Definition for singly-linked list.
# class ListNode(object):
#     def __init__(self, x):
#         self.val = x
#         self.next = None

# O(max(m,n))
class Solution(object):
    def addTwoNumbers(self, l1, l2):
        """
        :type l1: ListNode
        :type l2: ListNode
        :rtype: ListNode
        """
        carry = 0
        # dummy head
        head = curr = ListNode(0)
        while l1 or l2:
            val = carry
            if l1:
                val += l1.val
                l1 = l1.next
            if l2:
                val += l2.val
                l2 = l2.next
            curr.next = ListNode(val % 10)
            curr = curr.next
            carry = val / 10
        if carry > 0:
            curr.next = ListNode(carry)
        return head.next
\end{lstlisting}

C++:
\lstset{language=C++}
\begin{lstlisting}
class Solution {
public:
    ListNode *addTwoNumbers(ListNode *l1, ListNode *l2) {
        ListNode *res = new ListNode(0);
        ListNode *cur = res;
        int carry = 0;
        while (l1 || l2) {
            int n1 = l1 ? l1->val : 0;
            int n2 = l2 ? l2->val : 0;
            int sum = n1 + n2 + carry;
            carry = sum / 10;
            cur->next = new ListNode(sum % 10);
            cur = cur->next;
            if (l1) l1 = l1->next;
            if (l2) l2 = l2->next;
        }
        if (carry) cur->next = new ListNode(1);
        return res->next;
    }
};
\end{lstlisting}

\section{Add Two Numbers II (M)}
You are given two non-empty linked lists representing two non-negative integers. The most significant digit comes first and each of their nodes contain a single digit. Add the two numbers and return it as a linked list.\\

You may assume the two numbers do not contain any leading zero, except the number 0 itself.\\

Follow up:
What if you cannot modify the input lists? In other words, reversing the lists is not allowed. \\

Input: $(7 -> 2 -> 4 -> 3) + (5 -> 6 -> 4)$ \\
Output: $7 -> 8 -> 0 -> 7$\\

Python:
\lstset{language=python}
\begin{lstlisting}
# Definition for singly-linked list.
# class ListNode(object):
#     def __init__(self, x):
#         self.val = x
#         self.next = None

class Solution(object):
    def addTwoNumbers(self, l1, l2):
        """
        :type l1: ListNode
        :type l2: ListNode
        :rtype: ListNode
        """
        s1, s2 = [], []
        while l1:
            s1.append(l1.val)
            l1 = l1.next
        while l2:
            s2.append(l2.val)
            l2 = l2.next
        carry = 0
        curr = ListNode(0)
        while s1 or s2:
            val = carry
            if s1: val += s1.pop()
            if s2: val += s2.pop()
            curr.val = val % 10
            carry = val / 10
            head = ListNode(0)
            head.next = curr
            curr = head
        if carry > 0:
            curr.val = carry
            return curr
        else:
            return curr.next
\end{lstlisting}

\section{Plus One Linked List (E)}
Given a non-negative number represented as a singly linked list of digits, plus one to the number. The digits are stored such that the most significant digit is at the head of the list.\\

Example:\\
Input: $1->2->3$\\
Output: $1->2->4$\\

\begin{lstlisting}
class Solution {
public:
    ListNode *plusOne(ListNode *head) {
        if (!head)  return head;
        int carry = 1;
        ListNode *rev_head = reverse(head), *p = rev_head;
        
        while (p) {
            int sum = p->val + carry;
            if (sum == 10) {
                p->val = 0;
                carry = 1;
                p = p->next;
            } else {
                carry = 0;
                break;
            }
        }
        
        if (carry == 1) {
            p->next = new ListNode(1);
        }
        
        return revese(rev_head);
    }
    
    ListNode *reverse(ListNode *head) {
        ListNode *p1 = head->next, *p2;
        head->next = NULL;
        
        while (p1->next) {
            p2 = p1->next;
            p1->next = head;
            head = p1;
            p1 = p2;
        }
        
        return head;
    }
}; 
\end{lstlisting}


\section{Palindrome Linked List (E)}
Given a singly linked list, determine if it is a palindrome. \\

Python:
\lstset{language=python}
\begin{lstlisting}
# Definition for singly-linked list.
# class ListNode(object):
#     def __init__(self, x):
#         self.val = x
#         self.next = None

# stack
class Solution(object):
    def isPalindrome(self, head):
        """
        :type head: ListNode
        :rtype: bool
        """
        s = []
        x = head
        while x:
            s.append(x.val)
            x = x.next
        while s:
            tmp = s.pop()
            if tmp != head.val: return False
            head = head.next
        return True

# stack with slow and fast pointers
class Solution(object):
    def isPalindrome(self, head):
        """
        :type head: ListNode
        :rtype: bool
        """
        if not head: return True
        s = []
        slow, fast = head, head
        s.append(head.val)
        while fast.next and fast.next.next:
            slow = slow.next
            fast = fast.next.next
            s.append(slow.val)
        if not fast.next: s.pop()
        while slow.next:   
            slow = slow.next
            if s.pop() != slow.val: return False
        return True
\end{lstlisting} 

C++:
\lstset{language=C++}
\begin{lstlisting}
// reverse the right half list and compare it to the left half list
class Solution {
public:
    bool isPalindrome(ListNode *head) {
        if (!head || !head->next)   return true;
        
        ListNode *slow = head, *fast = head;
        while (fast->next && fast->next->next) {
            slow = slow->next;
            fast = fast->next->next;
        }
        
        slow->next = reverseList(slow->next);
        slow = slow->next;
        
        while (slow) {
            if (head->val != slow->val) return false;
            head = head->next;
            slow = slow->next;
        }
        
        return true;
    }
    
    ListNode *reverseList(ListNode *head) {
        ListNode *p1 = head->next, *p2;
        head->next = NULL;
        
        while (p1) {
            p2 = p1->next;
            p1->next = head;
            head = p1;
            p1 = p2;
        }
        
        return head;
    }
};
\end{lstlisting} 


\section{Swap Nodes in Pairs (E)}
Given a linked list, swap every two adjacent nodes and return its head. \\

For example, Given $1->2->3->4$, you should return the list as $2->1->4->3$. \\

Your algorithm should use only constant space. You may not modify the values in the list, only nodes itself can be changed. \\

\begin{lstlisting}
# Definition for singly-linked list.
# class ListNode(object):
#     def __init__(self, x):
#         self.val = x
#         self.next = None

# Iteration
class Solution(object):
    def swapPairs(self, head):
        """
        :type head: ListNode
        :rtype: ListNode
        """
        dummy = ListNode(0)
        dummy.next = head
        p = dummy
        while p.next and p.next.next:
            t = p.next.next
            p.next.next = t.next
            t.next = p.next
            p.next = t
            p = t.next
        return dummy.next
    
# Recursion
class Solution(object):
    def swapPairs(self, head):
        """
        :type head: ListNode
        :rtype: ListNode
        """
        if not head or not head.next:
            return head
        t = head.next
        head.next = self.swapPairs(head.next.next)
        t.next = head
        return t
\end{lstlisting}

\begin{lstlisting}
// iterative
class Solution {
public:
    ListNode *swapPairs(ListNode *head) {
        ListNode *new_head = new ListNode(0);
        new_head->next = head;
        ListNode *prev = new_head, *cur = head;
        
        while(cur && cur->next) {
            prev->next = cur->next;
            cur->next = cur->next->next;
            prev->next->next = cur;
            prev = cur;
            cur = cur->next;
        }
        
        return new_head->next;
    }
};

// recursive
class Solution {
public:
    ListNode* swapPairs(ListNode* head) {
        if (head == NULL || head->next == NULL)
            return head;
        
        ListNode *p = head->next;
        head->next = swapPairs(p->next);
        p->next = head;
        
        return p;
    }
};
\end{lstlisting}


\section{Reverse Linked List (E)}
Reverse a singly linked list.\\

\begin{lstlisting}
# Definition for singly-linked list.
# class ListNode(object):
#     def __init__(self, x):
#         self.val = x
#         self.next = None

# iteration with stack
class Solution(object):
    def reverseList(self, head):
        """
        :type head: ListNode
        :rtype: ListNode
        """
        s = []
        p = head
        while p:
            s.append(p.val)
            p = p.next
        dummy = ListNode(0)
        p = dummy
        while s:
            p.next = ListNode(s.pop())
            p = p.next
        return dummy.next
    
# iteration by inserting nodes after new head
class Solution(object):
    def reverseList(self, head):
        """
        :type head: ListNode
        :rtype: ListNode
        """
        newhead = None 
        while head:
            t = head.next
            head.next = newhead
            newhead = head
            head = t
        return newhead
        
# recursive
class Solution(object):
    def reverseList(self, head):
        """
        :type head: ListNode
        :rtype: ListNode
        """
        if not head or not head.next: return head
        newhead = self.reverseList(head.next)
        head.next.next = head
        head.next = None
        return newhead
\end{lstlisting}

\begin{lstlisting}
// recursive solution
class Solution {
public:   
    ListNode* reverseList(ListNode* head) {
        if (!head || !(head -> next)) 
            return head;
        ListNode* node = reverseList(head -> next);
        head -> next -> next = head;
        head -> next = NULL;
        return node; 
    }
}; 

// iterative solution
class Solution {
public:
    ListNode* reverseList(ListNode* head) {
        if (head == NULL || head->next == NULL)
            return head;
            
        ListNode *p1 = head->next, *p2;
        head->next = NULL;
        while (p1 != NULL) {
            p2 = p1->next;
            p1->next = head;
            head = p1;
            p1 = p2;
        }
        
        return head;
    }
};
\end{lstlisting} 


\section{Reverse Linked List II (M)}
Reverse a linked list from position m to n. Do it in-place and in one-pass.\\

For example:
Given $1->2->3->4->5->NULL$, m = 2 and n = 4,
return $1->4->3->2->5->NULL$.\\

Note:
Given m, n satisfy the following condition:
$1 \leq m \leq n \leq length of list$. \\

\begin{lstlisting}
# Definition for singly-linked list.
# class ListNode(object):
#     def __init__(self, x):
#         self.val = x
#         self.next = None

class Solution(object):
    def reverseBetween(self, head, m, n):
        """
        :type head: ListNode
        :type m: int
        :type n: int
        :rtype: ListNode
        """
        dummy = ListNode(0)
        dummy.next = head
        prev = dummy
        for i in range(m-1):
            prev = prev.next
        curr = prev.next
        for i in range(m, n):
            t = curr.next
            curr.next = t.next
            t.next = prev.next
            prev.next = t
        return dummy.next
\end{lstlisting}

\begin{lstlisting}
class Solution {
public:
    ListNode* reverseBetween(ListNode* head, int m, int n) {
        if (!head || m >= n) return head;
        ListNode *new_head = new ListNode(0);
        new_head->next = head;
        ListNode *pre = new_head;
        for (int i = 1; i < m; ++i) pre = pre->next;
        ListNode *cur = pre->next; // start of the mth node
        for (int i = 1; i <= n - m; ++i) {
            ListNode *tmp = cur->next;
            cur->next = tmp->next;
            tmp->next = pre->next;
            pre->next = tmp;
        }
        
        return new_head->next;
    }
};
\end{lstlisting}


\section{Rotate Linked List (M)}
Given a list, rotate the list to the right by k places, where k is non-negative.\\

For example:\\
Given $1->2->3->4->5->NULL$ and k = 2, return $4->5->1->2->3->NULL$.\\

\begin{lstlisting}
# Definition for singly-linked list.
# class ListNode(object):
#     def __init__(self, x):
#         self.val = x
#         self.next = None

class Solution(object):
    def rotateRight(self, head, k):
        """
        :type head: ListNode
        :type k: int
        :rtype: ListNode
        """
        if not head: return None
        n = 1
        cur = head
        while cur.next:
            cur = cur.next
            n += 1
        # make a cycle list
        cur.next = head
        # locate the element before the newhead after rotation
        m = n - k % n
        for i in range(m):
            cur = cur.next
        newhead = cur.next
        cur.next = None
        return newhead
\end{lstlisting}

\begin{lstlisting}
class Solution {
public:
    ListNode* rotateRight(ListNode* head, int k) {
        if (!head || k <= 0)  return head;
        ListNode *p = head;
        int n = 1;
        while (p->next){        // get the length of list
            p = p->next;
            ++n;
        }
        p->next = head;         // make a circle list
        k = k % n;              // get the correct k
        for(int i = 0; i < n - k; ++i)  // move p to the correct position
            p = p->next;
        head = p->next;
        p->next = NULL;
        return head;
    }
};
\end{lstlisting}


\section{Reorder List (M)}
Given a singly linked list L: $L0 -> L1 -> ... -> Ln-1 -> Ln$, reorder it to: $L0 -> Ln -> L1 -> Ln-1 -> L2 -> Ln-2 -> ...$ You must do this in-place without altering the nodes' values.\\

For example,
Given {1,2,3,4}, reorder it to {1,4,2,3}. \\

\begin{lstlisting}
// 1. Break the list to two lists from center
// 2. Reverse the second list
// 3. Insert the second list to the first list alternatively
class Solution {
public:
    void reorderList(ListNode* head) {
        if (!head || !head->next || !head->next->next) return;
        ListNode *slow = head, *fast = head;
        // step 1
        while (fast->next && fast->next->next) {
            slow = slow->next;
            fast = fast->next->next;
        }
        ListNode *mid = slow->next;
        slow->next = NULL;
        // step 2
        ListNode *new_head = reverseList(mid);
        // step 3
        while (head && new_head) {
            ListNode *tmp = head->next;
            head->next = new_head;
            new_head = new_head->next;
            head->next->next = tmp;
            head = tmp;
        }
    }
    ListNode *reverseList(ListNode *head) {
       ListNode *p1 = head->next, *p2;
       head->next = NULL;
       while (p1) {
           p2 = p1->next;
           p1->next = head;
           head = p1;
           p1 = p2;
       }
       return head;
    }
};
\end{lstlisting}


\section{Partition List (M)}
Given a linked list and a value x, partition it such that all nodes less than x come before nodes greater than or equal to x. You should preserve the original relative order of the nodes in each of the two partitions.\\

For example,
Given $1->4->3->2->5->2$ and x = 3,
return $1->2->2->4->3->5$. \\

\begin{lstlisting}
# Definition for singly-linked list.
# class ListNode(object):
#     def __init__(self, x):
#         self.val = x
#         self.next = None

class Solution(object):
    def partition(self, head, x):
        """
        :type head: ListNode
        :type x: int
        :rtype: ListNode
        """
        if not head: return head
        dummy1, dummy2 = ListNode(0), ListNode(0)
        dummy1.next = head
        p1, p2 = dummy1, dummy2
        while p1.next:
            if p1.next.val < x:
                p2.next = p1.next
                p1.next = p1.next.next
                p2 = p2.next
            else:
                p1 = p1.next
        p2.next = dummy1.next
        return dummy2.next
\end{lstlisting}

\begin{lstlisting}
class Solution {
public:
    ListNode* partition(ListNode* head, int x) {
        ListNode *head1 = new ListNode(0);
        ListNode *head2 = new ListNode(0);
        ListNode *p1 = head1;
        ListNode *p2 = head2;
        while (head) {
            if (head->val < x) {
                p1->next = head;
                p1 = p1->next;
            } else {
                p2->next = head;
                p2 = p2->next;
            }
            head = head->next;
        }
        p2->next = NULL;
        p1->next = head2->next;
        return head1->next;
    }
};
\end{lstlisting}


\section{Odd Even Linked List (M)}
Given a singly linked list, group all odd nodes together followed by the even nodes. Please note here we are talking about the node number and not the value in the nodes.\\

You should try to do it in place. The program should run in O(1) space complexity and O(nodes) time complexity.\\

Example:
Given $1->2->3->4->5->NULL$, return $1->3->5->2->4->NULL$.\\

Note:
The relative order inside both the even and odd groups should remain as it was in the input.
The first node is considered odd, the second node even and so on ... \\

\begin{lstlisting}
class Solution {
public:
    ListNode* oddEvenList(ListNode* head) {
        if (!head || !head->next) return head;
        ListNode *odd = head, *even = head->next;
        while (even && even->next) {
            ListNode *tmp = odd->next; // must store odd->next as tmp
            odd->next = even->next;
            even->next = even->next->next;
            odd->next->next = tmp; // update odd->next->next to tmp but not the current even
            even = even->next;
            odd = odd->next;
        }
        return head;
    }
};

class Solution {
public:
    ListNode* oddEvenList(ListNode* head) {
        if (!head || !head->next) return head;
        ListNode *odd = head, *even = head->next, *even_head = even;
        while (even && even->next) {
            odd->next = even->next;
            odd = even->next;
            even->next = odd->next;
            even = odd->next;
        }
        odd->next = even_head;
        return head;
    }

};
\end{lstlisting}


\section{Linked List Random Node (M)}
Given a singly linked list, return a random node's value from the linked list. Each node must have the same probability of being chosen.\\

Follow up:
What if the linked list is extremely large and its length is unknown to you? Could you solve this efficiently without using extra space?\\

Example:\\

// Init a singly linked list [1,2,3].\\
ListNode head = new ListNode(1);\\
head.next = new ListNode(2);\\
head.next.next = new ListNode(3);\\
Solution solution = new Solution(head);\\

// getRandom() should return either 1, 2, or 3 randomly. Each element should have equal probability of returning.\\
solution.getRandom();\\

\begin{lstlisting}
// 1. Regular solution 
class Solution {
public:
    /** @param head The linked list's head.
        Note that the head is guaranteed to be not null, so it contains at least one node. */
    Solution(ListNode* head) {
        n = 0;
        ListNode *p = head;
        this->head = head;
        while (p) {
            ++n; // get the length of list
            p = p->next;
        }
    }
    
    /** Returns a random node's value. */
    int getRandom() {
        int idx = rand() % n;
        ListNode *p = head;
        while (idx) {
            p = p->next;
            --idx;
        }
        return p->val;
    }
    
private:
    int n;
    ListNode *head;
};

// 2. Reservoir sampling
class Solution {
public:
    /** @param head The linked list's head. Note that the head is guanranteed to be not null, so it contains at least one node. */
    Solution(ListNode* head) {
        this->head = head;
    }
    
    /** Returns a random node's value. */
    int getRandom() {
        int res = head->val, i = 2;
        ListNode *cur = head->next;
        while (cur) {
            int j = rand() % i; // get a number from [0, i-1]
            if (j == 0) res = cur->val; // if j = 0, the random value is cur->val 
            ++i;
            cur = cur->next;
        }
        return res;
    }
private:
    ListNode *head;
};

/**
 * Your Solution object will be instantiated and called as such:
 * Solution obj = new Solution(head);
 * int param_1 = obj.getRandom();
 */
\end{lstlisting}


\section{Delete Node in a Linked List (E)}
Write a function to delete a node (except the tail) in a singly linked list, given only access to that node.\\

Supposed the linked list is 1-2-3-4 and you are given the third node with value 3, the linked list should become 1-2-4 after calling your function. \\

\begin{lstlisting}
# Definition for singly-linked list.
# class ListNode(object):
#     def __init__(self, x):
#         self.val = x
#         self.next = None

class Solution(object):
    def deleteNode(self, node):
        """
        :type node: ListNode
        :rtype: void Do not return anything, modify node in-place instead.
        """
        node.val = node.next.val
        t = node.next
        node.next = t.next
        del t
\end{lstlisting}

\begin{lstlisting}
class Solution {
public:
    void deleteNode(ListNode* node) {
        ListNode *tmp = node->next;
        *node = *tmp;
        delete tmp;
    }
};
\end{lstlisting} 


\section{Remove Linked List Elements (E)}
Remove all elements from a linked list of integers that have value val.\\

Example:\\
Given: 1-2-6-3-4-5-6, val = 6\\
Return: 1-2-3-4-5 \\

\begin{lstlisting}
# Definition for singly-linked list.
# class ListNode(object):
#     def __init__(self, x):
#         self.val = x
#         self.next = None

class Solution(object):
    def removeElements(self, head, val):
        """
        :type head: ListNode
        :type val: int
        :rtype: ListNode
        """
        dummy = ListNode(0)
        dummy.next = head
        prev, curr = dummy, head
        while curr:
            if curr.val == val:
                prev.next = curr.next  
                t = curr
                curr = curr.next
                del t
            else:
                prev = prev.next
                curr = curr.next
        return dummy.next
\end{lstlisting}

\begin{lstlisting}
class Solution {
public:
    ListNode* removeElements(ListNode* head, int val) {
        while (head != NULL && head->val == val)
            head = head->next;
            
        if (head == NULL)
            return head;
        
        ListNode *p1 = head, *p2 = head->next;
        while (p2 != NULL) {
            if (p2->val == val) {
                p1->next = p2->next;
            } else {
                p1 = p2;
            }
            p2 = p2->next;
        }
        
        return head;
    }
};
\end{lstlisting}


\section{Remove Nth Node From End of List (E)}
Given a linked list, remove the nth node from the end of list and return its head.\\

For example,\\
   Given linked list: $1->2->3->4->5$, and n = 2.\\
   After removing the second node from the end, the linked list becomes $1->2->3->5$.\\

\begin{lstlisting}
# Definition for singly-linked list.
# class ListNode(object):
#     def __init__(self, x):
#         self.val = x
#         self.next = None

class Solution(object):
    def removeNthFromEnd(self, head, n):
        """
        :type head: ListNode
        :type n: int
        :rtype: ListNode
        """
        dummy = ListNode(0)
        dummy.next = head
        p1, p2 = head, head
        for i in range(n):
            p2 = p2.next
        if not p2: 
            return p1.next
        while p2.next:
            p1 = p1.next
            p2 = p2.next
        p1.next = p1.next.next
        return dummy.next
\end{lstlisting}        

\begin{lstlisting}
class Solution {
public:
    ListNode* removeNthFromEnd(ListNode* head, int n) {
        ListNode* slow = head;
        ListNode* fast = head;
        
        for (int i = 1; i <= n; ++i)
            fast = fast->next;
            
        if (!fast)  return head->next;
        
        while (fast->next) {
            slow = slow->next;
            fast = fast->next;
        }
        slow->next = slow->next->next;      // remove the n-th node
        
        return head;
    }
};
\end{lstlisting}


\section{Remove Duplicates from Sorted List (E)}

Given a sorted linked list, delete all duplicates such that each element appear only once.\\ \\
For example: \\
Given 1-1-2, return 1-2. \\
Given 1-1-2-3-3, return 1-2-3. \\

\begin{lstlisting}
class Solution {
public:
    ListNode* deleteDuplicates(ListNode* head) {
        if (head == NULL || head->next == NULL)
            return head;
        
        ListNode *p1 = head, *p2 = head->next;
        
        while (p2 != NULL) {
            if (p1->val == p2->val) {
                p1->next = p2->next; 
            } else {
                p1 = p2;
            }
            p2 = p2->next;
        }
        
        return head;
    }
};
\end{lstlisting}


\section{Remove Duplicates from Sorted List II (M)}
Given a sorted linked list, delete all nodes that have duplicate numbers, leaving only distinct numbers from the original list.\\

For example,\\
Given 1-2-3-3-4-4-5, return 1-2-5.\\
Given 1-1-1-2-3, return 2-3. \\

\begin{lstlisting}
class Solution {
public:
    ListNode* deleteDuplicates(ListNode* head) {
        if (head == NULL || head->next == NULL)
            return head;
        
        ListNode *new_head = new ListNode(0);   // define a new head
        new_head->next = head;
        ListNode *p1 = new_head, *p2 = head;
        
        while(p2 != NULL) {
            while (p2->next != NULL && p2->val == p2->next->val)		// set p2 to the last node of duplicates
                p2 = p2->next;
            
            if (p1->next == p2) {
                p1 = p1->next;                // no duplicate
            } else {
                p1->next = p2->next;          // skip all duplicates
            }
            
            p2 = p2->next;
        }
        
        return new_head->next;
        
    }
};
\end{lstlisting}


\section{Linked List Cycle (E)}
Given a linked list, determine if it has a cycle in it. \\

\begin{lstlisting}
# Definition for singly-linked list.
# class ListNode(object):
#     def __init__(self, x):
#         self.val = x
#         self.next = None

class Solution(object):
    def hasCycle(self, head):
        """
        :type head: ListNode
        :rtype: bool
        """
        slow, fast = head, head
        while fast and fast.next:
            slow = slow.next
            fast = fast.next.next
            if slow == fast: return True
        return False
\end{lstlisting}            

\begin{lstlisting}
class Solution {
public:
    bool hasCycle(ListNode *head) {
        if (head == NULL)
            return false;
            
        ListNode *slow = head, *fast = head;
        while (fast->next != NULL && fast->next->next != NULL) {
            slow = slow->next;
            fast = fast->next->next;
            if (slow == fast)
                return true;
        }
        
        return false;
    }
};
\end{lstlisting}


\section{Linked List Cycle II (M)}
Given a linked list, return the node where the cycle begins. If there is no cycle, return null. \\
 
 \begin{lstlisting}
# Definition for singly-linked list.
# class ListNode(object):
#     def __init__(self, x):
#         self.val = x
#         self.next = None

class Solution(object):
    def detectCycle(self, head):
        """
        :type head: ListNode
        :rtype: ListNode
        """
        slow, fast = head, head
        while fast and fast.next:
            slow = slow.next
            fast = fast.next.next
            if slow == fast: break
        if not fast or not fast.next: return None
        slow = head
        while slow != fast:
            slow = slow.next
            fast = fast.next
        return slow
\end{lstlisting}
 
\begin{lstlisting}
/** 
 * len: length of the linked list
 * a: length of the head to the node where the cycle begins
 * b: length of the node where the cycle begins to the node where the slow and fast meet
 * r: length of the cycle
 * s: length of the slow moved
 * s = a + b, 2s = s + kr; -> a + b = kr = (k-1)r + r, r = len - a; -> a = (k-1)r + len - a - b
 */
class Solution {
public:
    ListNode *detectCycle(ListNode *head) {
        if (head == NULL)
            return NULL;

        ListNode *slow = head, *fast = head;
        
        while (fast->next != NULL && fast->next->next != NULL) {
            slow = slow->next;
            fast = fast->next->next;
            if (slow == fast) {
                fast = head;
                while (slow != fast) {
                    slow = slow->next;
                    fast = fast->next;
                }
                return slow;
            }
        }
        
        return NULL;
    }
};
\end{lstlisting}


\section{Merge Two Sorted Lists (E)}
Merge two sorted linked lists and return it as a new list. The new list should be made by splicing together the nodes of the first two lists.\\

\begin{lstlisting}
# Definition for singly-linked list.
# class ListNode(object):
#     def __init__(self, x):
#         self.val = x
#         self.next = None

class Solution(object):
    def mergeTwoLists(self, l1, l2):
        """
        :type l1: ListNode
        :type l2: ListNode
        :rtype: ListNode
        """
        dummy = ListNode(0)
        p = dummy
        while l1 and l2:
            if l1.val <= l2.val:
                p.next = l1
                l1 = l1.next
            else: 
                p.next = l2
                l2 = l2.next
            p = p.next
        if l1: p.next = l1
        if l2: p.next = l2
        return dummy.next
\end{lstlisting}
                
\begin{lstlisting}
class Solution {
public:
    ListNode* mergeTwoLists(ListNode* l1, ListNode* l2) {
        if (l1 == NULL) return l2;
        if (l2 == NULL) return l1;
        
        if (l1->val < l2->val) {
            l1->next = mergeTwoLists(l1->next, l2);
            return l1;
        } else {
            l2->next = mergeTwoLists(l2->next, l1);
            return l2;
        }
    }
};
\end{lstlisting}


\section{Merge k Sorted Lists (H)}
Merge k sorted linked lists and return it as one sorted list. Analyze and describe its complexity. \\
 
 \begin{lstlisting}
# Definition for singly-linked list.
# class ListNode(object):
#     def __init__(self, x):
#         self.val = x
#         self.next = None

# BF: Time - O(NlogN)  Space - O(N)
class Solution(object):
    def mergeKLists(self, lists):
        """
        :type lists: List[ListNode]
        :rtype: ListNode
        """
        p = dummy = ListNode(0)
        nodes = []
        for l in lists:
            while l:
                nodes.append(l.val)
                l = l.next
        for x in sorted(nodes):
            p.next = ListNode(x)
            p = p.next
        return dummy.next
 
# Priority queue: Time - O(NlogK), K is the number of linked lists
#                 Space - O(N)
from Queue import PriorityQueue
class Solution(object):
    def mergeKLists(self, lists):
        """
        :type lists: List[ListNode]
        :rtype: ListNode
        """
        p = dummy = ListNode(0)
        pq = PriorityQueue()
        # for each sublist, only save the first node to pq,
        # so the comparison complexity can be maintained as logK 
        # for insertion and pop.
        for l in lists:
            if l: 
                pq.put((l.val, l))
        while not pq.empty():
            val, node = pq.get()
            p.next = ListNode(val)
            p = p.next
            node = node.next
            if node: 
                pq.put((node.val, node))
        return dummy.next
\end{lstlisting}
 
\begin{lstlisting}
// Divide & Conquer
// Suppose initially each list is of average length n, then: k/2*(2n) + k/4*(4n) + k/8*(8n)... + = logk * (kn)
class Solution {
public:
    ListNode *mergeKLists(vector<ListNode *> &lists) {
        if(lists.empty())   return NULL;
        
        while(lists.size() > 1){
            lists.push_back(mergeTwoLists(lists[0], lists[1]));
            lists.erase(lists.begin());
            lists.erase(lists.begin());
        }
        return lists.front();
    }
    
    ListNode* mergeTwoLists(ListNode* l1, ListNode* l2) {
        if (l1 == NULL) return l2;
        if (l2 == NULL) return l1;
        
        if (l1->val < l2->val) {
            l1->next = mergeTwoLists(l1->next, l2);
            return l1;
        } else {
            l2->next = mergeTwoLists(l2->next, l1);
            return l2;
        }
    }
};
\end{lstlisting}


\section{Flatten Nested List Iterator (M)}
Given a nested list of integers, implement an iterator to flatten it. Each element is either an integer, or a list -- whose elements may also be integers or other lists.\\

Example 1: Given the list [[1,1],2,[1,1]],\\
By calling next repeatedly until hasNext returns false, the order of elements returned by next should be: [1,1,2,1,1].\\

Example 2: Given the list [1,[4,[6]]],\\
By calling next repeatedly until hasNext returns false, the order of elements returned by next should be: [1,4,6]. \\
 
\begin{lstlisting}
/**
 * // This is the interface that allows for creating nested lists.
 * // You should not implement it, or speculate about its implementation
 * class NestedInteger {
 *   public:
 *     // Return true if this NestedInteger holds a single integer, rather than a nested list.
 *     bool isInteger() const;
 *
 *     // Return the single integer that this NestedInteger holds, if it holds a single integer
 *     // The result is undefined if this NestedInteger holds a nested list
 *     int getInteger() const;
 *
 *     // Return the nested list that this NestedInteger holds, if it holds a nested list
 *     // The result is undefined if this NestedInteger holds a single integer
 *     const vector<NestedInteger> &getList() const;
 * };
 */
 
class NestedIterator {
public:
    NestedIterator(vector<NestedInteger> &nestedList) {
        for (int i = nestedList.size() - 1; i >= 0; --i)
            s.push(nestedList[i]);
    }

    int next() {
        NestedInteger n = s.top();
        s.pop();
        return n.getInteger();
    }

    bool hasNext() {
        while(!s.empty()) {
            NestedInteger n = s.top();
            if (n.isInteger())  return true;
            s.pop();    // pop the current element for the following push
            // push to s if the current element is a list
            for (int i = n.getList().size() - 1; i >= 0; --i)   
                s.push(n.getList()[i]);
        }
        return false;
    }
    
private:
    stack<NestedInteger> s;
};

/**
 * Your NestedIterator object will be instantiated and called as such:
 * NestedIterator i(nestedList);
 * while (i.hasNext()) cout << i.next();
 */
\end{lstlisting}


\section{Nested List Weight Sum (E)}
Given a nested list of integers, return the sum of all integers in the list weighted by their depth. Each element is either an integer, or a list -- whose elements may also be integers or other lists. \\

Example 1:\\
Given the list [[1,1],2,[1,1]], return 10. (four 1's at depth 2, one 2 at depth 1)\\

Example 2:\\
Given the list [1,[4,[6]]], return 27. (one 1 at depth 1, one 4 at depth 2, and one 6 at depth 3; 1 + 4*2 + 6*3 = 27) \\
 
\begin{lstlisting}
class Solution {
public:
    int depthSum(vector<NestedInteger>& nestedList) {
        int res = 0;
        for (auto l : nestedList)
            res += getSum(l, 1);
        return res;
    }
        
    int getSum(NestedInteger nl, int level) {
        int res = 0;
        if (nl.isInteger())
            return level * nl.getInteger();
        for (auto l : nl.getList())
            res += getSum(l, level+1);
        return res;
    }
};
\end{lstlisting}


\section{Nested List Weight Sum II (M)}
Given a nested list of integers, return the sum of all integers in the list weighted by their depth. Each element is either an integer, or a list -- whose elements may also be integers or other lists.\\

Different from the previous question where weight is increasing from root to leaf, now the weight is defined from bottom up. i.e., the leaf level integers have weight 1, and the root level integers have the largest weight. \\

Example 1:\\
Given the list [[1,1],2,[1,1]], return 8. (four 1's at depth 1, one 2 at depth 2)\\

Example 2:\\
Given the list [1,[4,[6]]], return 17. (one 1 at depth 3, one 4 at depth 2, and one 6 at depth 1; 1*3 + 4*2 + 6*1 = 17)  \\

\begin{lstlisting}
class Solution {
public:
    int depthSumInverse(vector<NestedInteger> &nestedList) {
        int unweighted = 0, weighted = 0;
        
        while (!nestedList.empty()) {
            vector<NestedInteger> nextLevel;
            for (auto l : nestedList) {
                if (l.isInteger()) {
                    unweighted += l.getInteger();       // add all integers of the current level togeter
                } else {
                    // insert(position, first, last)
                    nextLevel.insert(nextLevel.end(), l.getList().begin(), l.getList().end());
                }
            }
            weighted += unweighted;     // add all integers in level n to level n+1 twice
            nestedList = nextLevel;     // update the nested list
        }
        
        return weighted;
    }
};
\end{lstlisting}


\section{Mini Parser (M)}
Given a nested list of integers represented as a string, implement a parser to deserialize it. Each element is either an integer, or a list -- whose elements may also be integers or other lists.\\

Note: You may assume that the string is well-formed:\\
    String is non-empty.\\
    String does not contain white spaces.\\
    String contains only digits 0-9, [, -, ,, ].\\

Example 1:
Given s = "324",
You should return a NestedInteger object which contains a single integer 324.\\

Example 2:
Given s = "[123,[456,[789]]]",
Return a NestedInteger object containing a nested list with 2 elements:\\
1. An integer containing value 123.\\
2. A nested list containing two elements:\\
    i.  An integer containing value 456.\\
    ii. A nested list with one element:\\
         a. An integer containing value 789.\\

\begin{lstlisting}
/**
 * // This is the interface that allows for creating nested lists.
 * // You should not implement it, or speculate about its implementation
 * class NestedInteger {
 *   public:
 *     // Constructor initializes an empty nested list.
 *     NestedInteger();
 *
 *     // Constructor initializes a single integer.
 *     NestedInteger(int value);
 *
 *     // Return true if this NestedInteger holds a single integer, rather than a nested list.
 *     bool isInteger() const;
 *
 *     // Return the single integer that this NestedInteger holds, if it holds a single integer
 *     // The result is undefined if this NestedInteger holds a nested list
 *     int getInteger() const;
 *
 *     // Set this NestedInteger to hold a single integer.
 *     void setInteger(int value);
 *
 *     // Set this NestedInteger to hold a nested list and adds a nested integer to it.
 *     void add(const NestedInteger &ni);
 *
 *     // Return the nested list that this NestedInteger holds, if it holds a nested list
 *     // The result is undefined if this NestedInteger holds a single integer
 *     const vector<NestedInteger> &getList() const;
 * };
 */
// 1. Recursive solution
class Solution {
public:
    NestedInteger deserialize(string s) {
        if (s.empty()) return NestedInteger();
        if (s[0] != '[') return NestedInteger(stoi(s));
        if (s.size() <= 2) return NestedInteger();
        NestedInteger res;
        int start = 1, cnt = 0;
        for (int i = 1; i < s.size(); ++i) {
            if (cnt == 0 && (s[i] == ',' || i == s.size() - 1)) {
                res.add(deserialize(s.substr(start, i - start)));
                start = i + 1;
            } else if (s[i] == '[') ++cnt;
            else if (s[i] == ']') --cnt;
        }
        return res;
    }
};

// 2. Iterative solution
class Solution {
public:
    NestedInteger deserialize(string s) {
        if (s.empty()) return NestedInteger();
        if (s[0] != '[') return NestedInteger(stoi(s));
        stack<NestedInteger> st;
        int start = 1;
        for (int i = 0; i < s.size(); ++i) {
            if (s[i] == '[') {
                st.push(NestedInteger());
                start = i + 1;
            } else if (s[i] == ',' || s[i] == ']') {
                if (i > start) {
                    st.top().add(NestedInteger(stoi(s.substr(start, i - start))));
                }
                start = i + 1;
                if (s[i] == ']') {
                    if (st.size() > 1) {
                        NestedInteger t = st.top(); st.pop();
                        st.top().add(t);
                    }
                }
            }
        }
        return st.top();
    }
};

// 3. istringstream solution
class Solution {
public:
    NestedInteger deserialize(string s) {
        istringstream in(s);
        return deserialize(in);
    }
    NestedInteger deserialize(istringstream& in) {
        int num;
        if (in >> num) return NestedInteger(num);
        in.clear();
        in.get();
        NestedInteger list;
        while (in.peek() != ']') {
            list.add(deserialize(in));
            if (in.peek() == ',') {
                in.get();
            }
        }
        in.get();
        return list;
    }
};
\end{lstlisting}

