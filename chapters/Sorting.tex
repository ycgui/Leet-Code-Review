\chapter{Sorting}
\section{Sort Colors (M)}
Given an array with n objects colored red, white or blue, sort them so that objects of the same color are adjacent, with the colors in the order red, white and blue. Here, we will use the integers 0, 1, and 2 to represent the color red, white, and blue respectively.\\

Note: You are not suppose to use the library's sort function for this problem.\\

Follow up:\\
A rather straight forward solution is a two-pass algorithm using counting sort.\\
First, iterate the array counting number of 0's, 1's, and 2's, then overwrite array with total number of 0's, then 1's and followed by 2's.\\
Could you come up with an one-pass algorithm using only constant space?\\

\begin{lstlisting}
// 1. Count and Sort
class Solution {
public:
    void sortColors(vector<int> &nums) {
        int count[3] = {0}, idx = 0;
        // Count the number of each color
        for (int i = 0; i < nums.size(); ++i) ++count[nums[i]];
        // Sort the array by assigning correct number of colors based on counts
        for (int i = 0; i < 3; ++i) {
            for (int j = 0; j < count[i]; ++j) {
                nums[idx++] = i;
            }
        }
    }
};

// 2. Two pointers
class Solution {
public:
    void sortColors(vector<int> &nums) {
        int red = 0, blue = nums.size()-1, i = 0;
        while (i < blue + 1) {
            if (nums[i] == 0) swap(nums[i++], nums[red++]);
            // nums[i] still need to be checked in the next loop after swap
            else if (nums[i] == 2) swap(nums[i], nums[blue--]);
            else ++i;
        }
    }
};
\end{lstlisting}


\section{Wiggle Sort (M)}
Given an unsorted array nums, reorder it in-place such that $nums[0] <= nums[1] >= nums[2] <= nums[3]....$\\

For example, given nums = [3, 5, 2, 1, 6, 4], one possible answer is [1, 6, 2, 5, 3, 4].\\

\begin{lstlisting}
// Wiggle Sort O(nlogn)
class Solution {
public:
    void wiggleSort(vector<int>& nums) {
        sort(nums.begin(), nums.end());
        if (nums.size() <= 2)   return;
        for (int i = 2; i < num.size(); i += 2) {
            swap(nums[i], nums[i-1]);
        }
    }
};


// Wiggle Sort O(n)
class Solution {
public:
    void wiggleSort(vector<int>& nums) {
        if (nums.size() <= 1)   return;
        for (int i = 1; i < nums.size(); ++i) {
            if ( (i % 2 == 1 && nums[i] < nums[i-1]) ||
                 (i % 2 == 0 && nums[i] > nums[i-1]) ) {
                    swap(nums[i], nums[i-1]);
                 }
        }
    }
};
\end{lstlisting}


\section{Wiggle Sort II (M)}
Given an unsorted array nums, reorder it such that $nums[0] < nums[1] > nums[2] < nums[3]....$ \\

Example:\\
(1) Given nums = [1, 5, 1, 1, 6, 4], one possible answer is [1, 4, 1, 5, 1, 6]. \\
(2) Given nums = [1, 3, 2, 2, 3, 1], one possible answer is [2, 3, 1, 3, 1, 2].\\

Note: You may assume all input has valid answer.\\

Follow Up: Can you do it in O(n) time and/or in-place with O(1) extra space?\\

\begin{lstlisting}
// Wiggle Sort II Time: O(nlogn) Space: O(n)
class Solution {
public:
    void wiggleSort(vector<int>& nums) {
        vector<int> tmp = nums;
        int n = nums.size(), k = (n+1)/2, j = n;
        sort(tmp.begin(), tmp.end());
        for (int i = 0; i < n; ++i) {
            if (i % 2 == 0) {
                nums[i] = tmp[--k];
            } else {
                nums[i] = tmp[--j];
            }
        }
    }
};

// O(1) space
class Solution {
public:
    void wiggleSort(vector<int>& nums) {
        #define A(i) nums[(1 + 2 * i) % (n | 1)]
        int n = nums.size(), i = 0, j = 0, k = n - 1;
        auto midptr = nums.begin() + n / 2;
        nth_element(nums.begin(), midptr, nums.end());
        int mid = *midptr;
        while (j <= k) {
            if (A(j) > mid) swap(A(i++), A(j++));
            else if (A(j) < mid) swap(A(j), A(k--));
            else ++j;
        }
    }
};
\end{lstlisting}


\section{Wiggle Subsequence (M)}
A sequence of numbers is called a wiggle sequence if the differences between successive numbers strictly alternate between positive and negative. The first difference (if one exists) may be either positive or negative. A sequence with fewer than two elements is trivially a wiggle sequence.\\

For example, [1,7,4,9,2,5] is a wiggle sequence because the differences (6,-3,5,-7,3) are alternately positive and negative. In contrast, [1,4,7,2,5] and [1,7,4,5,5] are not wiggle sequences, the first because its first two differences are positive and the second because its last difference is zero.\\

Given a sequence of integers, return the length of the longest subsequence that is a wiggle sequence. A subsequence is obtained by deleting some number of elements (eventually, also zero) from the original sequence, leaving the remaining elements in their original order.\\

Examples:\\

Input: [1,7,4,9,2,5]\\
Output: 6\\
The entire sequence is a wiggle sequence.\\

Input: [1,17,5,10,13,15,10,5,16,8]\\
Output: 7\\
There are several subsequences that achieve this length. One is [1,17,10,13,10,16,8].\\

Input: [1,2,3,4,5,6,7,8,9]\\
Output: 2\\

Follow up: Can you do it in O(n) time? \\

\begin{lstlisting}
// O(n^2)
class Solution {
public:
    int wiggleMaxLength(vector<int>& nums) {
        if (nums.empty()) return 0;
        int n = nums.size();
        vector<int> p(n, 1);
        vector<int> q(n, 1);
        
        for (int i = 1; i < n; ++i) {
            for (int j = 0; j < i; ++j) {
                if (nums[i] > nums[j]) {
                    p[i] = max(p[i], q[j] + 1);
                } else if (nums[i] < nums[j]) {
                    q[i] = max(q[i], p[j] + 1);
                } else {
                    continue;
                }
            }
        }
        
        return max(p.back(), q.back());
    }
};

// O(n)
class Solution {
public:
    int wiggleMaxLength(vector<int>& nums) {
        int p = 1, q = 1, n = nums.size();
        
        for (int i = 1; i < n; ++i) {
            if (nums[i] > nums[i - 1]){ 
                p = q + 1;
            } else if (nums[i] < nums[i - 1]) {
                q = p + 1;
            }
        }
        
        return min(n, max(p, q));
    }
};
\end{lstlisting}


\section{Sort List (M)}
Sort a linked list in O(n log n) time using constant space complexity.\\

\begin{lstlisting}
// Merge Sort
class Solution {
public:
    ListNode *sortList(ListNode *head) {
        if (!head || !head->next) return head;
        ListNode *p1 = head, *p2 = head;
        while (p2->next && p2->next->next) {
            p1 = p1->next;
            p2 = p2->next->next;
        }
        p2 = sortList(p1->next);
        p1->next = NULL;
        p1 = sortList(head);
        return mergeList(p1, p2);
    }
    ListNode *mergeList(ListNode *p1, ListNode *p2) {
        ListNode *new_head = new ListNode(0);
        ListNode *tmp = new_head;
        while (p1 && p2) {
            if (p1->val < p2->val) {
                tmp->next = p1;
                p1 = p1->next;
            } else {
                tmp->next = p2;
                p2 = p2->next;
            }
            tmp = tmp->next;
        }
        if (p1) tmp->next = p1;
        if (p2) tmp->next = p2;
        return new_head->next;
    }
};
\end{lstlisting}


\section{Insertion Sort List (M)}
Sort a linked list using insertion sort.\\

\begin{lstlisting}
class Solution {
public:
    ListNode* insertionSortList(ListNode* head) {
        if (!head || !head->next) return head;
        ListNode *new_head = new ListNode(0);
        ListNode *cur, *next;
        while (head) {
            cur = new_head; // reset cur to the beginning for each iteration
            next = head->next; // next point to the next position of head
            // search the correct position to insert head
            while (cur->next && cur->next->val <= head->val) {
                cur = cur->next;
            }
            head->next = cur->next; // add head after cur
            cur->next = head; // update cur->next
            head = next; // update head
        }
        return new_head->next;
    }
};
\end{lstlisting}





