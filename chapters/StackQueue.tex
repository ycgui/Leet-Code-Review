\chapter{Stack and Queue}


\section{Implement Stack using Queues (E)}
Implement the following operations of a stack using queues.\\

    push(x) -- Push element x onto stack.\\
    pop() -- Removes the element on top of the stack.\\
    top() -- Get the top element.\\
    empty() -- Return whether the stack is empty.\\

\begin{lstlisting}
class MyStack(object):

    def __init__(self):
        """
        Initialize your data structure here.
        """
        self.queue = []

    def push(self, x):
        """
        Push element x onto stack.
        :type x: int
        :rtype: None
        """
        self.queue.insert(0, x)
        n = len(self.queue)
        for i in range(1, n):
            self.queue.insert(0, self.queue[-1])
            self.queue.pop()
        

    def pop(self):
        """
        Removes the element on top of the stack and returns that element.
        :rtype: int
        """
        return self.queue.pop()
        

    def top(self):
        """
        Get the top element.
        :rtype: int
        """
        return self.queue[-1]
        

    def empty(self):
        """
        Returns whether the stack is empty.
        :rtype: bool
        """
        return not self.queue
        

# Your MyStack object will be instantiated and called as such:
# obj = MyStack()
# obj.push(x)
# param_2 = obj.pop()
# param_3 = obj.top()
# param_4 = obj.empty()
\end{lstlisting}

\begin{lstlisting}
class Stack {
queue<int> que;
public:
    // Push element x onto stack.
    void push(int x) {
       que.push(x);                             // push x to tail
       for (int i = 1; i < que.size(); ++i) {   // repeat until x is the head
           que.push(que.front());               // push head to tail
           que.pop();                           // pop the old head
       }
    }

    // Removes the element on top of the stack.
    void pop() {
        que.pop();
    }

    // Get the top element.
    int top() {
        return que.front();
    }

    // Return whether the stack is empty.
    bool empty() {
        return que.empty();
    }
};
\end{lstlisting}


\section{Implement Queue using Stacks (E)}
 Implement the following operations of a queue using stacks.\\

    push(x) -- Push element x to the back of queue.\\
    pop() -- Removes the element from in front of queue.\\
    peek() -- Get the front element.\\
    empty() -- Return whether the queue is empty.\\
  
 \begin{lstlisting}  
 class MyQueue(object):

    def __init__(self):
        """
        Initialize your data structure here.
        """
        self.s1 = []
        self.s2 = []
        

    def push(self, x):
        """
        Push element x to the back of queue.
        :type x: int
        :rtype: None
        """
        while self.s1:
            self.s2.append(self.s1.pop())
        self.s1.append(x)
        while self.s2:
            self.s1.append(self.s2.pop())
        

    def pop(self):
        """
        Removes the element from in front of queue and returns that element.
        :rtype: int
        """
        return self.s1.pop()
        

    def peek(self):
        """
        Get the front element.
        :rtype: int
        """
        return self.s1[-1]
        

    def empty(self):
        """
        Returns whether the queue is empty.
        :rtype: bool
        """
        return not self.s1

# Your MyQueue object will be instantiated and called as such:
# obj = MyQueue()
# obj.push(x)
# param_2 = obj.pop()
# param_3 = obj.peek()
# param_4 = obj.empty()   
\end{lstlisting}

\begin{lstlisting}
class Queue {
stack<int> s1, s2;
public:
    // Push element x to the back of queue.
    void push(int x) {
        while (!s2.empty()){
            s1.push(s2.top());
            s2.pop();
        }
        s1.push(x);
    }

    // Removes the element from in front of queue.
    void pop(void) {
        while (!s1.empty()) {
            s2.push(s1.top());
            s1.pop();
        }
        s2.pop();
    }

    // Get the front element.
    int peek(void) {
        while (!s1.empty()) {
            s2.push(s1.top());
            s1.pop();
        }
        return s2.top();
    }

    // Return whether the queue is empty.
    bool empty(void) {
        return s1.empty() && s2.empty();
    }
};
\end{lstlisting}


\section{Min Stack (E)}
 Design a stack that supports push, pop, top, and retrieving the minimum element in constant time.\\

    push(x) -- Push element x onto stack.\\
    pop() -- Removes the element on top of the stack.\\
    top() -- Get the top element.\\
    getMin() -- Retrieve the minimum element in the stack.\\
    
Example:\\
MinStack minStack = new MinStack();\\
minStack.push(-2);\\
minStack.push(0);\\
minStack.push(-3);\\
minStack.getMin();   $->$ Returns -3.\\
minStack.pop();\\
minStack.top();      $->$ Returns 0.\\
minStack.getMin();   $->$ Returns -2.\\

\begin{lstlisting}
class MinStack {
public:
    stack<int> s;
    stack<int> s_min;                       // use a new stack to store minimum numbers
    
    void push(int x) {
        s.push(x);
        if (s_min.empty() || x <= getMin()) // make a copy of a min value to s_min
            s_min.push(x);
    }
    
    void pop() {
        if (s.top() == getMin())            // update s_min if the min value is poped
            s_min.pop();
        s.pop();
    }
    
    int top() {
        return s.top();
    }
    
    int getMin() {
        return s_min.top();
    }
};
\end{lstlisting}


\section{Moving Average from Data Stream (E)}
Given a stream of integers and a window size, calculate the moving average of all integers in the sliding window.\\

For example,\\
MovingAverage m = new MovingAverage(3);\\
m.next(1) = 1\\
m.next(10) = (1 + 10) / 2\\
m.next(3) = (1 + 10 + 3) / 3\\
m.next(5) = (10 + 3 + 5) / 3 \\

\begin{lstlisting}
class MovingAverage {
public:
    MovingAverage(int size) {
        this->size = size;
        sum = 0;
    }
    
    double next(int val) {
        if (q.size() >= size) {	
            sum -= q.front();
            q.pop();
        }
        q.push(val);
        sum += val;
        return sum / q.size();
    }
    
private:
    queue<int> q;
    int size;
    double sum;
};
\end{lstlisting}


\section{Evaluate Reverse Polish Notation (M)}
Evaluate the value of an arithmetic expression in Reverse Polish Notation. Valid operators are +, -, *, /. Each operand may be an integer or another expression.\\

Some examples:\\
  $["2", "1", "+", "3", "*"] -> ((2 + 1) * 3) -> 9$ \\
  $["4", "13", "5", "/", "+"] -> (4 + (13 / 5)) -> 6$ \\

\begin{lstlisting}
// atoi: string to integer
// c_str(): Returns a pointer to a null-terminated character array with data equivalent to those stored in the string
class Solution {
public:
    int evalRPN(vector<string> &tokens) {
        if (tokens.size() == 1) return atoi(tokens[0].c_str());
        stack<int> s;
        for (int i = 0; i < tokens.size(); ++i) {
            if (tokens[i] != "+" && tokens[i] != "-" && tokens[i] != "*" && tokens[i] != "/") {
                s.push(atoi(tokens[i].c_str()));
            } else {
                int m = s.top();
                s.pop();
                int n = s.top();
                s.pop();
                if (tokens[i] == "+") s.push(n + m);
                if (tokens[i] == "-") s.push(n - m);
                if (tokens[i] == "*") s.push(n * m);
                if (tokens[i] == "/") s.push(n / m);
            }
        }
        return s.top();
    }
};
\end{lstlisting}

\section{Remove K Digits (M)}
Given a non-negative integer num represented as a string, remove k digits from the number so that the new number is the smallest possible.\\

Note:\\

    The length of num is less than 10002 and will be $>=$ k.\\
    The given num does not contain any leading zero.\\

Example 1:\\
Input: num = "1432219", k = 3\\
Output: "1219"\\
Explanation: Remove the three digits 4, 3, and 2 to form the new number 1219 which is the smallest.\\

Example 2:\\
Input: num = "10200", k = 1\\
Output: "200"\\
Explanation: Remove the leading 1 and the number is 200. Note that the output must not contain leading zeroes.\\

Example 3:\\
Input: num = "10", k = 2\\
Output: "0"\\
Explanation: Remove all the digits from the number and it is left with nothing which is 0.\\

\begin{lstlisting}
class Solution(object):
    def removeKdigits(self, num, k):
        """
        :type num: str
        :type k: int
        :rtype: str
        """
        res = []
        # only leave the small digits in res
        for n in num:
            while k > 0 and res and res[-1] > n:
                res.pop()
                k -= 1
            res.append(n)
        # make sure exact k digits are removed
        while k > 0 and res:
            res.pop()
            k -= 1
        # remove leading 0s
        while res and res[0] == '0':
            res.pop(0)
        # join list to resume str
        if res:
            return ''.join(res)
        else:
            return '0'
\end{lstlisting}

\section{Kth Largest Element in an Array (M)}
Find the kth largest element in an unsorted array. Note that it is the kth largest element in the sorted order, not the kth distinct element.\\

For example,
Given [3,2,1,5,6,4] and k = 2, return 5.\\

Note:
You may assume k is always valid, $1 \leq k \leq array's length$.\\

\begin{lstlisting}
# O(NlogN)
class Solution(object):
    def findKthLargest(self, nums, k):
        """
        :type nums: List[int]
        :type k: int
        :rtype: int
        """
        nums.sort()
        return nums[len(nums)-k]
        
# O(Nlogk)
from Queue import PriorityQueue
class Solution(object):
    def findKthLargest(self, nums, k):
        """
        :type nums: List[int]
        :type k: int
        :rtype: int
        """
        pq = PriorityQueue()
        # O(Nlogk)
        for num in nums:
            pq.put(num)
            if pq.qsize() == k + 1: pq.get()
        return pq.get()
\end{lstlisting}

\section{Top K Frequent Elements (M)}
Given a non-empty array of integers, return the k most frequent elements.\\

For example,
Given [1,1,1,2,2,3] and k = 2, return [1,2].\\

Note:
    You may assume k is always valid, $1 \leq k \leq $ number of unique elements.\\
    Your algorithm's time complexity must be better than O(n log n), where n is the array's size.\\

\begin{lstlisting}
from Queue import PriorityQueue
class Solution(object):
    def topKFrequent(self, nums, k):
        """
        :type nums: List[int]
        :type k: int
        :rtype: List[int]
        """
        pq = PriorityQueue()
        n = len(nums)
        freq_map, res = {}, []
        # O(N)
        for i in range(n):
            if nums[i] in freq_map:
                freq_map[nums[i]] += 1
            else:
                freq_map.update({nums[i]: 1})
        # O(Mlogk), M is the number of element in freq_map
        for num, freq in freq_map.items():
            pq.put((freq, num))
            if pq.qsize() == k + 1: pq.get()
        # O(klogk)
        for i in range(k):
            freq, num = pq.get()
            res.append(num)
        return res
\end{lstlisting}        
        
\begin{lstlisting}
class Solution {
public:
    vector<int> topKFrequent(vector<int>& nums, int k) {
        unordered_map<int, int> m;
        // Priority queues are a type of container adaptors, specifically designed such that its first element is always the greatest of the elements it contains, according to some strict weak ordering criterion.
        priority_queue<pair<int, int>> q;
        vector<int> res;
        for (auto a : nums) {
            ++m[a];
        }
        for (auto it : m) {
            q.push({it.second, it.first});
        }
        for (int i = 0; i < k; ++i) {
            res.push_back(q.top().second); 
            q.pop();
        }
        return res;
    }
};
\end{lstlisting}

\section{Kth Smallest Element in a Sorted Matrix (M)}
Given a n x n matrix where each of the rows and columns are sorted in ascending order, find the kth smallest element in the matrix.\

Note that it is the kth smallest element in the sorted order, not the kth distinct element.\\

Example:\\
matrix = [
   [ 1,  5,  9],
   [10, 11, 13],
   [12, 13, 15]
],\\
k = 8, return 13.\\

Note:
You may assume k is always valid, $1 \leq k \leq n^2$.\\

\begin{lstlisting}
from Queue import PriorityQueue
class Solution(object):
    def kthSmallest(self, matrix, k):
        """
        :type matrix: List[List[int]]
        :type k: int
        :rtype: int
        """
        pq = PriorityQueue()
        m, n = len(matrix), len(matrix[0])
        res = []
        for i in range(m):
            for j in range(n):
                pq.put(matrix[i][j])
        for i in range(m * n):
            num = pq.get()
            res.append(num)
        return res[k-1]
\end{lstlisting}


\begin{lstlisting}
// 1. Binary search
class Solution {
public:
    int kthSmallest(vector<vector<int>>& matrix, int k) {
        int left = matrix[0][0], right = matrix.back().back();
        while (left < right) {
            int mid = left + (right - left) / 2, cnt = 0;
            // upper_bound(): Returns an iterator pointing to the first element 
            // in the range [first,last) which compares greater than val.
            for (int i = 0; i < matrix.size(); ++i) {
                // get the position of upper_bound compared to mid
                cnt += upper_bound(matrix[i].begin(), matrix[i].end(), mid) - matrix[i].begin();
            }
            if (cnt < k) left = mid + 1;
            else right = mid;
        }
        return left;
    }
};

// 2. Heap
class Solution {
public:
    int kthSmallest(vector<vector<int>>& matrix, int k) {
        priority_queue<int, vector<int>> q;
        for (int i = 0; i < matrix.size(); ++i) {
            for (int j = 0; j < matrix[i].size(); ++j) {
                q.emplace(matrix[i][j]);
                if (q.size() > k) q.pop();
            }
        }
        return q.top();
    }
};
\end{lstlisting}

\section{K Closest Points to Origin (Facebook Onsite 2019.6.28)}
We have a list of points on the plane.  Find the K closest points to the origin (0, 0).

(Here, the distance between two points on a plane is the Euclidean distance.)

You may return the answer in any order.  The answer is guaranteed to be unique (except for the order that it is in.)

 

Example 1:

Input: points = [[1,3],[-2,2]], K = 1
Output: [[-2,2]]
Explanation: 
The distance between (1, 3) and the origin is sqrt(10).
The distance between (-2, 2) and the origin is sqrt(8).
Since sqrt(8) < sqrt(10), (-2, 2) is closer to the origin.
We only want the closest K = 1 points from the origin, so the answer is just [[-2,2]].

Example 2:

Input: points = [[3,3],[5,-1],[-2,4]], K = 2
Output: [[3,3],[-2,4]]
(The answer [[-2,4],[3,3]] would also be accepted.)

 

Note:

    1 <= K <= points.length <= 10000
    -10000 < points[i][0] < 10000
    -10000 < points[i][1] < 10000
    
\begin{lstlisting}
from Queue import PriorityQueue
import math
class Solution(object):
    def kClosest(self, points, K):
        """
        :type points: List[List[int]]
        :type K: int
        :rtype: List[List[int]]
        """
        res = []
        pq = PriorityQueue()
        for p in points:
            d = math.sqrt(p[0]**2 + p[1]**2)
            pq.put((d, p))
        for i in range(K):
            dist, point = pq.get()
            res.append(point)
        return res
\end{lstlisting}
                


