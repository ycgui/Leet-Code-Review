\chapter{Stack and Queue}


\section{Implement Stack using Queues (E)}
Implement the following operations of a stack using queues.\\

    push(x) -- Push element x onto stack.\\
    pop() -- Removes the element on top of the stack.\\
    top() -- Get the top element.\\
    empty() -- Return whether the stack is empty.\\

\begin{lstlisting}
class Stack {
queue<int> que;
public:
    // Push element x onto stack.
    void push(int x) {
       que.push(x);                             // push x to tail
       for (int i = 1; i < que.size(); ++i) {   // repeat until x is the head
           que.push(que.front());               // push head to tail
           que.pop();                           // pop the old head
       }
    }

    // Removes the element on top of the stack.
    void pop() {
        que.pop();
    }

    // Get the top element.
    int top() {
        return que.front();
    }

    // Return whether the stack is empty.
    bool empty() {
        return que.empty();
    }
};
\end{lstlisting}


\section{Implement Queue using Stacks (E)}
 Implement the following operations of a queue using stacks.\\

    push(x) -- Push element x to the back of queue.\\
    pop() -- Removes the element from in front of queue.\\
    peek() -- Get the front element.\\
    empty() -- Return whether the queue is empty.\\
    
\begin{lstlisting}
class Queue {
stack<int> s1, s2;
public:
    // Push element x to the back of queue.
    void push(int x) {
        while (!s2.empty()){
            s1.push(s2.top());
            s2.pop();
        }
        s1.push(x);
    }

    // Removes the element from in front of queue.
    void pop(void) {
        while (!s1.empty()) {
            s2.push(s1.top());
            s1.pop();
        }
        s2.pop();
    }

    // Get the front element.
    int peek(void) {
        while (!s1.empty()) {
            s2.push(s1.top());
            s1.pop();
        }
        return s2.top();
    }

    // Return whether the queue is empty.
    bool empty(void) {
        return s1.empty() && s2.empty();
    }
};
\end{lstlisting}


\section{Min Stack (E)}
 Design a stack that supports push, pop, top, and retrieving the minimum element in constant time.\\

    push(x) -- Push element x onto stack.\\
    pop() -- Removes the element on top of the stack.\\
    top() -- Get the top element.\\
    getMin() -- Retrieve the minimum element in the stack.\\
    
Example:\\
MinStack minStack = new MinStack();\\
minStack.push(-2);\\
minStack.push(0);\\
minStack.push(-3);\\
minStack.getMin();   $->$ Returns -3.\\
minStack.pop();\\
minStack.top();      $->$ Returns 0.\\
minStack.getMin();   $->$ Returns -2.\\

\begin{lstlisting}
class MinStack {
public:
    stack<int> s;
    stack<int> s_min;                       // use a new stack to store minimum numbers
    
    void push(int x) {
        s.push(x);
        if (s_min.empty() || x <= getMin()) // make a copy of a min value to s_min
            s_min.push(x);
    }
    
    void pop() {
        if (s.top() == getMin())            // update s_min if the min value is poped
            s_min.pop();
        s.pop();
    }
    
    int top() {
        return s.top();
    }
    
    int getMin() {
        return s_min.top();
    }
};
\end{lstlisting}


\section{Moving Average from Data Stream (E)}
Given a stream of integers and a window size, calculate the moving average of all integers in the sliding window.\\

For example,\\
MovingAverage m = new MovingAverage(3);\\
m.next(1) = 1\\
m.next(10) = (1 + 10) / 2\\
m.next(3) = (1 + 10 + 3) / 3\\
m.next(5) = (10 + 3 + 5) / 3 \\

\begin{lstlisting}
class MovingAverage {
public:
    MovingAverage(int size) {
        this->size = size;
        sum = 0;
    }
    
    double next(int val) {
        if (q.size() >= size) {	
            sum -= q.front();
            q.pop();
        }
        q.push(val);
        sum += val;
        return sum / q.size();
    }
    
private:
    queue<int> q;
    int size;
    double sum;
};
\end{lstlisting}


\section{Evaluate Reverse Polish Notation (M)}
Evaluate the value of an arithmetic expression in Reverse Polish Notation. Valid operators are +, -, *, /. Each operand may be an integer or another expression.\\

Some examples:\\
  $["2", "1", "+", "3", "*"] -> ((2 + 1) * 3) -> 9$ \\
  $["4", "13", "5", "/", "+"] -> (4 + (13 / 5)) -> 6$ \\

\begin{lstlisting}
// atoi: string to integer
// c_str(): Returns a pointer to a null-terminated character array with data equivalent to those stored in the string
class Solution {
public:
    int evalRPN(vector<string> &tokens) {
        if (tokens.size() == 1) return atoi(tokens[0].c_str());
        stack<int> s;
        for (int i = 0; i < tokens.size(); ++i) {
            if (tokens[i] != "+" && tokens[i] != "-" && tokens[i] != "*" && tokens[i] != "/") {
                s.push(atoi(tokens[i].c_str()));
            } else {
                int m = s.top();
                s.pop();
                int n = s.top();
                s.pop();
                if (tokens[i] == "+") s.push(n + m);
                if (tokens[i] == "-") s.push(n - m);
                if (tokens[i] == "*") s.push(n * m);
                if (tokens[i] == "/") s.push(n / m);
            }
        }
        return s.top();
    }
};
\end{lstlisting}




